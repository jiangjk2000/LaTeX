\documentclass[device=pad,lang=cn,UTF8]{elegantbook}

\title{马原复习}
\subtitle{马克思主义基本原理概论}
\author{$\aleph$}
\institute{Jiang Xi Science and Technology University}
\date{\today}
\version{1.00}
\extrainfo{Victory won\rq t come to us unless we go to it. --- M. Moore}
\logo{kls}
\cover{1599}

\begin{document}

    \maketitle
    \tableofcontents
    \mainmatter
    \hypersetup{pageanchor=true}
    \chapter{世界的物质性及发展规律}
    \begin{enumerate}
        \item 意识对物质具有反作用
        \begin{itemize}
            \item 意识活动具有目的性和计划性:人在认识客观世界、尊重客观规律的同时,还总是根据一定的目的和要求去确定反映什么、不反映什么,以及怎样反映,从而表现出主体的选择性。
            \item 意识活动具有创造性:人的意识不仅采取感觉、知觉、表象等形式,反映事物的外部现象,而且运用概念、判断、推理等形式,对感性材料进行加工制作和选择建构,在思维中构造一个现实中所没有的理想世界。
            \item 意识具有指导实践改造客观世界的作用:意识的能动作用不限于从实践中形成一定的思想,形成活动的目的、计划、方法等观念的东西,更重要的在于以这些观念的东西为指导,通过实践使之一步步变为客观现实。
            \item 意识具有调控人的行为和生理活动的作用:现代科学和医学实验证明:意识、心理因素能够对人的行为选择和健康状况产生重要影响。
        \end{itemize}
        \item 必然性与偶然性的概念及辩证关系
        \begin{itemize}
            \item 必然性是指事物联系和发展中一定要发生的、不可避免的趋势。偶然性是指事物联系和发展中不确定的趋向。必然性和偶然性是对立统一的关系。
            \item 二者是对立的,它们是事物发展的两种不同趋向,产生的原因以及在事物发展中的地位和作用不同。
            \item 二者是统一的,其表现是:
            \begin{itemize}
                \item 必然性总是通过大量的偶然性表现出来,由此为自己开辟道路,没有脱离偶然性的纯粹必然性;
                \item 偶然性是必然性的表现形式和必要补充,偶然性的背后暗藏着必然性并受其制约,没有脱离必然性的纯粹偶然性;
                \item 必然性和偶然性可以在一定条件下互相转化。
            \end{itemize}
            \item 必然性和偶然性辩证关系的原理,对指导科学研究和社会实践有重大意义。
        \end{itemize}
        \item 矛盾的特殊性原理\par
        矛盾的特殊性是指矛盾着的事物及其每一个侧面各有其特点。矛盾是普遍存在的,但不同事物的矛盾又各不相同,都有其特殊性。\par
        其特殊性有三种形式:
        \begin{itemize}
            \item 不同事物的矛盾各有其特点。
            \item 同一事物的矛盾在不同发展过程和发展阶段各有不同的特点。
            \item 构成事物的诸多矛盾以及每一矛盾的不同方面各有不同的性质,地位和作用。
        \end{itemize}
    \end{enumerate}

    \chapter{实践与认识及其发展规律}
    \begin{enumerate}
        \item 感性认识与理性认识的概念及辩证关系
        \begin{description}
            \item[感性认识] 感性认识是人们在实践基础上,由感觉器官直接感受到的关于事物的表现、事物的外部联系、事物的各个方面的认识,包括感觉、知觉和表象的三种形式。
            \item[理性认识] 理性认识是指人们借助抽象思维,在概括整理大量感性材料的基础上,达到关于事物的本质、全体、内部联系和事物自身规律性的认识。理性认识包括概念、判断和推理三种形式。
        \end{description}
        感性认识和理性认识的性质虽然不同,但又不是互相分离的,它们的辩证关系可以概括为:
        \begin{itemize}
            \item 感性认识有待于发展和深化为理性认识;
            \item 理性认识依赖于感性认识;
            \item 感性认识和理性认识相互渗、相互包含,两者在实践的基础上统一起来。
        \end{itemize}
        \item 实践是认识的基础
        \begin{itemize}
            \item 实践是认识的来源。
            \item 实践是认识发展的的动力,认识是随实践的发展而发展的。
            \item 实践是检验认识正确与否的唯一标准。
            \item 实践是认识的目的,人们认识世界的目的在于指导实践有效地改造世界。
        \end{itemize}
        \item 真理的绝对性和相对性的含义及相互关系
        \begin{itemize}
            \item 真理的绝对性:是指真理主客观统一的确定性和发展的无限性。
            \begin{itemize}
                \item 一是指任何真理都标志着主观与客观之间的符合,都包含着不依赖于人的意识的客观内容,都同谬误有原则的界限。
                \item 二是人类认识按其本性来说,能够正确认识无限发展着的世界,认识前进每一步,都是对无限发展着的物质世界的接近。
            \end{itemize}
            \item 真理的相对性:是指人们在一定条件下对客观事物及其本质和发展规律的正确认识总是有限度的,不完善的。从客观世界的整体来看,任何真理都只是对客观世界的某一阶段,某一部分的正确认识,人类已经达到的认识的广度总是有限的。就特定事物而言,任何真理都只是对客观对象一定方面,一定层次和一定程度的正确认识。
            \item 真理的绝对性和相对性是辩证统一的。二者相互依存,人们对于客观事物及其本质和规律的每一个正确认识,都是在一定范围内,一定程度上,一定条件下的认识,因而都是相对的,有局限的。
            二者相互包含,真理的绝对性寓于真理的相对性之中。真理的相对性包含着真理的绝对性。
        \end{itemize}
    \end{enumerate}
    \chapter{人类社会及其发展规律}
    \begin{enumerate}
        \item 科学技术的两面性
        \begin{itemize}
            \item 科学技术像一把双刃剑,既能通过促进经济和社会发展以造福人类,同时也可能在一定条件下对人类的生存和发展带来消极后果。
            \item 科学技术的发展标志着人类改造自然能力的增强,意味着能够更多地创造出人们所需的物质财富,对社会发展的积极作用是主要的、基本的方面。但是,由于对科学技术应用不当等原因,也会产生一定的消极后果。一种情形是由于对自然规律和人与自然的关系认识不够,或缺乏对科学技术消极后果的强有力的控制手段而产生的。还有一种与一定的社会制度有关。在资本主义条件下,科学技术的发展并非都能使人们摆脱贫困,并非都能促进人们的身心健康发展。科学技术有时“表现为异己的、敌对的和统治的权利”。世界上的霸权主义者利用现代科技发展武器,入侵他国,造成大量生命财产的损失,就是一个例证。
        \end{itemize}
        \item 人民群众在创造历史过程中的决定作用
        \begin{itemize}
            \item 人民群众是社会历史的主体,是历史的创造者。这是马克思主义最基本的观点之一。\par
            从本质上看,人民群众是指一切对社会历史发展起推动作用的人;从量上看,人民群众是指社会人口中的绝大多数。
            \item 在社会历史发展过程中,人民群众起着决定性的作用。人民群众是社会历史实践的主体。
            \item 在创造历史中起决定性的作用。
            \item 人民群众是社会物质财富的创造者\par
            广大的劳动群众是物质资料生产活动的主体,创造了人们吃穿住行等必需的生活资料以及从事政治、科学、文化艺术等活动所必需的物质前提。
            \item 人民群众是社会精神财富的创造者
            人民群众通过物质生产实践为创造精神财富提供了必要的物质条件和设施。人民群众的生活、实践活动是一切精神财富、精神产品形成和发展的源泉。人民群众还直接参与了社会精神财富的创造。
            \item 人民群众是社会变革的决定力量\par
            人民群众在创造社会财富的同时,也创造并改造着社会关系。生产关系的变革,社会制度的更替,最终取决于生产力的发展,但不会随着生产力的发展自发地实现和完成,而必须借助人民群众的力量。\par
            人民群众创造历史的活动受到一定社会历史条件的制约,我国的社会主义制度为人民群众创造历史的活动提供了极为有利的经济、政治和精神文化等方面的条件,但也存在有待完善和改进的方面。
        \end{itemize}
        \item 英雄史观\par
        英雄史观是历史唯心主义的典型表现,是马克思主义对英雄史观本质的揭示。英雄史观否认人民群众是创造历史的决定力量,把英雄看作历史的主宰。这里说的“英雄”,是指个别杰出人物,主要是指帝王将相和少数英雄人物,他们或者具有非凡的才智(主观唯心主义观点),或者是某种神秘的客观精神的代表(客观唯心主义观点),他们的思想动决定历史的发展。\par
        一切唯心史观都有两个基本缺陷:
        \begin{itemize}
            \item 考察了人们历史活动的思想动机,但没有探究产生这些动机的原因,未能揭示社会发展的经济根源,这就是用社会意识解释社会存在,否认物质生产是社会存在和发展的基础。
            \item 未能正确说明人民群众在创造历史中的决定作用。这两方面是相互联系的。
        \end{itemize}
        英雄史观正是从社会意识决定社会存在这一唯心史观的根本观点出发,否认物质生产的历史作用,进而否认以物质生产者为主体的人民群众是历史发展的决定力量,把英雄的意志、思想、才智看作历史发展的最终动力。由于社会历史条件的限制、阶级的偏见和认识论上的原因,英雄史观在社会历史领域中长期占据统治地位。只有历史唯物主义才第一次科学地阐明社会发展的经济根源,正确估计人民群众和个人在历史发展中的作用。英雄史观是和历史唯物主义相对立的典型的历史唯心主义。
    \end{enumerate}
    \chapter{资本主义的本质及规律}
    \begin{enumerate}
        \item 剩余价值的含义及其产生
        \begin{itemize}
            \item 剩余价值是在资本主义的生产过程中生产出来的,它是由雇佣工人的剩余劳动创造出来的,它即不是有全部资本创造的,也不是由不变资本创造的,而是又可变资本雇佣的劳动者创造的。
            \item 对剩余价值的生产方法有两种:绝对剩余价值的生产和相对剩余价值的生产。绝对剩余价值的生产指在必要的劳动时间不变的条件下,由于延长工作日的长度和提高劳动强度而生产的剩余价值。相对剩余价值指工作日长度不变的情况下,通过缩短劳动时间而相对延长剩余劳动时间产生的剩余价值。
        \end{itemize}
        \item 资本原始积累和资本剥削
        \begin{itemize}
            \item 资本原始积累指通过暴力使直接生产者与生产资料相分离,由此使货币财富迅速集中于少数人手中的历史过程。这个过程发生在资本及与之相适应的生产方式形成前的历史阶段,所以称为“原始积累”。它是资本主义生产方式的前提和起点,对农民土地的剥夺,形成整个原始积累的基础。
            \item 资本剥削的实质:就是资本家无偿地占有由工人的劳动所产生的剩余价值。\par
            其实区别在于,前者是赤裸裸、血腥和暴力获取资本,后者是将原始积累的资本转化为自身发展的资本,如压榨劳动者劳动力,压缩成本、扩大规模,加快自身工业进程之类获取更多资本的过程。            
        \end{itemize}
        \item 价值规律及其作用
        \begin{itemize}
            \item 价值规律是商品生产和商品交换的基本规律。这一规律的主要内容和客观要求是:商品的价值量由生产商品的社会必要劳动时间决定,商品交换以价值量为基础,按照等价交换的原则进行。\par
            在商品经济中,价值规律的表现形式是,商品的价格围绕商品的价值自发波动。
            \item 价值规律的作用表现在:
            \begin{itemize}
                \item 自发地调节生产资料和劳动力在社会各生产部门之间的分配比例。
                \item 自发地刺激社会生产力的发展。在商品经济条件下,商品是按照由社会必要劳动时间所决定的社会价值进行交换的。
                \item 自发地调节社会收入的分配。
            \end{itemize}
            \item 价值规律在对经济活动进行自发调节时,会产生一些消极的后果:
            \begin{itemize}
                \item 导致社会资源浪费。
                \item 阻碍技术的进步。
                \item 导致收入两极分化。
            \end{itemize}
        \end{itemize}
        \item 不变资本、可变资本和剩余价值率的含义
        \begin{itemize}
            \item 根据资本在剩余价值生产过程中的作用不同,资本可以区分为不变资本和可变资本。\par
            不变资本是以生产资料形态存在的资本,它通过工人的具体劳动转移到新产品中去,其价值量不会大于它原有的价值量。以生产资料形式存在的资本,在生产过程中只转移自己的物质形态而不改变自己的价值量,不发生增殖,因此称之为不变资本。\par
            可变资本是用来购买劳动力的那部分资本,在生产过程中不是被转移到新产品中去,而是由工人的劳动再生产出来。再生产过程中,工人所创造的新价值,不仅包括相当于劳动力价值的价值,而且还包括一定量的剩余价值,因此称之为可变资本。
            \item 剩余价值率是在资本主义条件下,是工人受资本家剥削程度的表现。公式中,~$m'$~为剩余价值率,~$m$~为剩余价值,~$v$~为可变资本。年剩余价值率是一年内生产的剩余价值总量和预付可变资本的比率。
        \end{itemize}
    \end{enumerate}
    \chapter{资本主义发展及其趋势}
    \begin{enumerate}
        \item 资本主义的历史地位
        \begin{itemize}
            \item 资本主义在历史上首起过巨大的革命作用\par
            资本主义制度取代封建制度是社会发展中的一次大飞跃。它全面破坏了封建主义的社会关系和意识形态,在社会生活的各个领域中引起了——系列革命性的变革。资本主义不仅消灭了封建割据状态,建立厂统一的国家,而且消灭了古老的民族工业,许多国家实现了工业化和生产的商品化、社会化,开拓了世界市场。\par
            资本主义把世界变成丁一个开放的世界,使—切国家的生产和消费都成为世界性的了,使各国经济形成了相互依赖,渗透和竞争的新格局。
            \item 资本主义的每一个进步都包含着自己的反面\par
            资本主义生产关系的建立,适应于生产力发展的需要。但是它是以一种私有制代替了另—种私有制,以—种剥削制度代替另—种剥削制度。资本主义私有制和建立在它基础上的雇佣劳动制度,是产生自己一切反面的根源。\par
            资本主义提供了造福人类、解放人类的物质条件,但却利用这种物质条件破坏了人类的进一步解放,把人类又推向灾难的深渊。\par
            总之,资本主义在繁荣进步的之后,包含着日益加深的社会资本主义向更高社会形态转化的必然性。
        \end{itemize}
        \item 社会主义代替资本主义是一个长期的历史过程\par
        资本主义必然为社会主义所代替,并不意味着资本主义社会将在短期内自行消亡。资本主义制度目前还能为生产力的发展提供一定的空间。它向社会主义的转变会触及资产阶级的根本利益,必然会遭到阻挠和反抗,因而资本主义向社会主义的过渡必然是一个复杂的、长期的历史过程。首先,任何社会形态的存在都有相对稳定性,从产生到衰亡都要经过相当长的时间跨度。\par
        其次,资本主义发展的不平衡性决定了过渡的长期性。最后,当代资本主义的发展,还显示出生产关系对生产力容纳的空间,说明资本主义为社会主义所代替尚需长期的过程。但必须明确,尽管资本主义在全世界被社会主义所取代是一个相当长的历史过程,并且这个过程可能出现这样那样的曲折,但资本主义为社会主义所取代的总趋势,则是必然的历史走向。马克思在论述资本主义生产方式时指出:“发展社会劳动的生产力,是资本的历史任务和存在理由。资本正是以此不自觉地创造着一种更高级的生产形式的物质条件。”这是对资本主义历史过渡性最精辟而辩证的论述。
    \end{enumerate}
    \chapter{共产主义崇高理想及其最终实现}
    \begin{enumerate}
        \item 实现共产主义是长期的历史过程\par
        实现共产主义是一个长期的历史过程,这是因为:
        \begin{itemize}
            \item 实现共产主义需要在实践中长期探索;
            \item 社会主义向共产主义过渡是一个长期过程;
            \item 经济落后国家实现共产主义需要更长的实践过程;
            \item 共产主义在全世界范围的实现是长期、曲折的过程。
        \end{itemize}
        \item 共产主义理想实现的必然性
        \begin{itemize}
            \item 共产主义理想实现的历史必然性:
            \begin{itemize}
                \item 共产主义理想的实现,是以人类社会发展规律以及资本主义社会的基本矛盾发展为依据的;
                \item 社会主义运动的实践,特别是社会主义国家的兴起和不断发展,已经并正在用事实证明着共产主义理想实现的必然性。
            \end{itemize}
            \item 马克思主义不仅从社会形态交替规律上对共产主义理想实现的必然性作了一般性的历史观论证,而且通过对资本主义社会的具体剖析,作了具体实证的证明。证明了:
            \begin{itemize}
                \item 资本主义发展的自我否定的趋势;
                \item 揭示了资本主义生产社会化与生产资料私人占有的基本矛盾,论证了资本主义的历史暂时性;
                \item 揭示了资本主义剥削的秘密,证明了资本主义的非正义性,论证了工人阶级推翻旧世界建设新世界的历史使命;
                \item 揭示了工人阶级和资产阶级斗争的发展规律和趋势,论述了工人阶级解放斗争胜利的必然性。
            \end{itemize}
            \item 当然,共产主义理想实现的必然性,并不意味着实现共产主义理想的过程是一帆风顺的,不是一个长期的过程。
        \end{itemize}
    \end{enumerate}
\end{document}