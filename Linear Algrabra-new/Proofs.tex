\documentclass[lang=cn,11pt,normal]{elegantbook}
\usepackage{mathtools}
\usepackage{LAbf}
\usepackage{hyperref}
% title info
\title{Proofs}
\subtitle{定理,定义,推论的证明}
% bio info
\author{JJK}
\institute{Jiang Xi science and technology University}
\date{\today}
% extra info
\version{1.00}
\extrainfo{Even though the truth is not a happy time, we have to be honest ,because more effort needed to cover the truth. --- Bertr Russell}
\logo{logo.png}
\cover{cover2.jpg}
\begin{document}
	\maketitle
	\tableofcontents
	\mainmatter
	\hypersetup{pageanchor,bookmarksopen=true}

	\chapter{数学分析}
	\section{多元函数微分学}
	\subsection{偏导与微分}
	\begin{definition}{Language余项}{}
			设$D\subset \RR^n$是一个凸区域$f \in C^{m+1} (D),\bfa=(a_1,\cdots,a_n),\bfa+\bfh=(a_1+h_1,\cdots,a_n+h_n)$是$D$中的两个点,则必存在$\theta \in (0,1),$使得
			\begin{equation}
			f(\boldsymbol{a+h})=\sum^m_{k=0}\sum_{|\bfa|=k} \frac{\mathbf{D}^{\boldsymbol{\alpha}}f(\bfa)}{\boldsymbol{\alpha}!} \boldsymbol{h^\alpha} + \RR_m,
			\end{equation}
			
			其中
			\begin{gather}
			\RR_m=\sum_{|\boldsymbol{\alpha}|=m+1} \frac{\mathbf{D}^{\boldsymbol{\alpha}}f(\bfa+\theta \bfh)}{\boldsymbol{\alpha}!} \boldsymbol{h^\alpha},
			\end{gather}
			称为\boldmath{Lagrange}余项.
	\end{definition}
	\begin{proof}
		固定$\bfa$和$\bfh$,设$t\in[0,1]$,考虑$[0,1]$上的函数$\varphi(t)=f(\bfa+t\bfh)$.显然$\varphi$在$[0,1]$上有$m+1$阶的连续导数.对$\varphi$用单变量函数的Taylor公式,得
		\begin{equation}
		\varphi(1)=\varphi (0)+\varphi^{'}(0)+\frac{1}{2!}\varphi^{''}(0)+\cdots+\frac{1}{m!}\varphi^{(m)}(0)+\frac{1}{(m+1)!}\varphi^{(m+1)}(\theta),
		\end{equation}
		其中$\theta \in (0,1).$\\
		显然$\varphi(1)=f(\boldsymbol{a+h}),\varphi(0)=f(\bfa)$.根据复合函数的求导公式,得\\
		\begin{equation}
		\varphi^{'}(t)=\frac{\partial f}{\partial x_1}(\bfa+t\bfh)h_1+\dots+\frac{\partial f}{\partial x_n}(\bfa+t\bfh)h_n
		\end{equation}
		\begin{equation}
		=\left(h_1\frac{\partial}{\partial x_1}+\dots+h_n\frac{\partial}{\partial x_n}\right)f(\bfa+t\bfh).
		\end{equation}
		这就是说,对$\varphi$求一次导数,相当于把算子
		\begin{equation}
		h_1\frac{\partial}{\partial x_1}+\dots+h_n\frac{\partial}{\partial x_n}
		\end{equation}
		作用于函数$f(\bfa+t\bfh)$.因而如9.9节中的例5所说的那样,有
		\begin{equation}
		\varphi^{''}(t)=\left(h_1\frac{\partial}{\partial x_1}+\dots+h_n\frac{\partial}{\partial x_n}\right)^2f(\bfa+t\bfh),
		\end{equation}
		\begin{equation}
		\vdots
		\end{equation}
		\begin{equation}
		\varphi^{(m)}(t)=\left(h_1\frac{\partial}{\partial x_1}+\dots+h_n\frac{\partial}{\partial x_n}\right)^{m}f(\bfa+t\bfh).
		\end{equation}
		根据引理9.10.1,得
		\begin{equation}
		\varphi^{(k)}(t)=\left(h_1\frac{\partial}{\partial x_1}+\dots+h_n\frac{\partial}{\partial x_n}\right)^{k}f(\bfa+t\bfh)
		\end{equation}
		\begin{equation}
		=\sum_{|\boldsymbol{\alpha}|=k}\frac{k!}{\boldsymbol{\alpha}!}\frac{\partial^{\alpha_1}}{\partial x^{\alpha_1}_1}\cdots\frac{\partial^{\alpha_n}}{\partial x^{\alpha_n}_n}f(\bfa+t\bfh)\boldsymbol{h^{\alpha}}
		\end{equation}
		\begin{equation}
		=\sum_{|\boldsymbol{\alpha}|=k}\frac{k!}{\boldsymbol{\alpha}!}\mathbf{D}^{\boldsymbol{\alpha}}f(\bfa+t\bfh)\boldsymbol{h^a}.
		\end{equation}
		所以\\
		\begin{equation}
		\varphi^{k}(0)=\frac{k!}{\boldsymbol{\alpha}!}\mathbf{D}^{\boldsymbol{\alpha}}f(\bfa)\boldsymbol{h^a}
		\end{equation}
		将式(1.13)代入式(1.3),即得要证明的式(1.1).
	\end{proof}
	\chapter{高等代数}
	\section{第一章}
	\subsection{一元多项式}
	\begin{definition}{带余除法}{}
		设$f(x),g(x)\in \mathbb{F}[x]$,其中$g(x)\ne0$,则一定存在唯一的$q(x),r(x)\in\mathbb{F}[x]$,使
		\begin{equation}
		f(x)=q(x)g(x)+r(x)\label{2.1}
		\end{equation}
		成立,其中$deg(r(x))<deg(g(x))$,或者$r(x)=0$.
	\end{definition}
	\begin{proof}
		(\ref{2.1})式中与的存在性可以对进行数学归纳来证明,略.\\
		下证唯一性.\\
		设如果另有$q'(x),r'(x)\in\mathbb{F}[x]$,使
		\begin{equation}
		f(x)=q'(x)g(x)+r'(x)
		\end{equation}
		成立,其中$deg(r'(x))<deg(g(x))$或者$r'(x)=0$.于是
		\begin{equation}
			q(x)g(x)+r(x)=q'(x)g(x)+r'(x),
		\end{equation}
		所以
		\begin{equation}
			(q(x)-q'(x))g(x)=r'(x)-r(x).
		\end{equation}
		如果$q(x)\ne q'(x)$,又因为$g(x)\ne0$,则$r'(x)-r(x)\ne 0$,并且
		\begin{equation}
			deg(r'(x)-r(x))=deg(q(x)-q'(x))+deg(g(x))\geqslant deg(g(x))
		\end{equation}
		这一矛盾证明了$q(x)=q'(x)$,从而$r'(x)=r(x)$.
	\end{proof}
	\begin{theorem}{ }{ }
		 设$f(x),g(x)\in \mathbb{F}[x]$,其中$g(x)\ne 0$,则$g(x)$整除$f(x)$的充分必要条件是$g(x)$除$f(x)$的余式为0.
	\end{theorem}
	\begin{proof}
		如果$r(x)=0$,则$f(x)=q(x)g(x)$,即$g(x)|f(x)$.反过来,如果$g(x)|f(x)$,则存在$q(x)\in\mathbb{F}[x]$,使得$f(x)=q(x)g(x)=q(x)g(x)+0$,即$r(x)=0$.
	\end{proof}
	\subsection{线性方程组}
	\subsubsection{数域}
	\begin{definition}{数域}{}
		复数集的一个子集$K$如果满足:\\
			$
			(1)0,1\in K;\\
			(2)a,b\in K\rightarrow a\pm b,ab\in K,\\
			a,b\in K,$且$b\ne 0\rightarrow \frac{a}{b}\in K,
			$\\
			那么,称$K$是一个数域.
	\end{definition}
\end{document}