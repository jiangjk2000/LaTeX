\documentclass[lang=cn,11pt,normal]{elegantbook}
\usepackage{mathtools}
% title info
\title{Linear Algrabra}
\subtitle{高等代数作业}
% bio info
\author{JJK}
\institute{Jiang Xi science and technology University}
\date{\today}
% extra info
\version{1.00}
\extrainfo{Even though the truth is not a happy time, we have to be honest ,because more effort needed to cover the truth. --- Bertr Russell}
\logo{logo.png}
\cover{cover.jpg}

\begin{document}
	\maketitle
	\mainmatter
	\hypersetup{pageanchor=true}
	% add preface chapter here if needed
	\chapter{高等代数}
	\section{作业}
	\begin{exercise}
		用正交变换化二次型$f(x_1,x_2,x_3)=x_1^2-2x_2^2-2x_3^2-4x_1x_2+4x_1x_3+8x_2x_3$为标准型,并给出所用的正交变换.
	\end{exercise}
	\begin{proof}
		这个实二次型的矩阵$\boldsymbol{A}$为\\
		$$
		\boldsymbol{A}=
		\begin{bmatrix}
		1&-2&2\\
		-2&-2&4\\
		2&4&-2
		\end{bmatrix}.
		$$
		$$
		|\lambda\boldsymbol{I}-\boldsymbol{A}|=
		\begin{vmatrix}
		\lambda-1&2&-2\\
		2&\lambda+2&-4\\
		-2&2-4&\lambda+2
		\end{vmatrix}
		=(\lambda-2)^2(\lambda+7).
		$$
		$\boldsymbol{A}$的全部特征值是2(二重),-7.\\
		对特征值2,求出$(2\boldsymbol{I}-\boldsymbol{A})\boldsymbol{X}=0$的基础解系:
		$$
		\boldsymbol{\alpha}_1=
		\begin{bmatrix}
		2\\0\\1
		\end{bmatrix},
		\boldsymbol{\alpha}_2=
		\begin{bmatrix}
		-2\\-1\\0
		\end{bmatrix}.
		$$
		对特征值-7,求出$(-7\boldsymbol{I}-\boldsymbol{A})\boldsymbol{X}=0$的基础解系:\\
		$$
		\boldsymbol{\alpha}_3=
		\begin{bmatrix}
		-1\\-2\\2
		\end{bmatrix}.
		$$
		得到三个线性无关的特征向量
		$$
		\boldsymbol{\alpha}_1=
		\begin{bmatrix}
		2\\0\\1
		\end{bmatrix},
		\boldsymbol{\alpha}_2=
		\begin{bmatrix}
		-2\\-1\\0
		\end{bmatrix},
		\boldsymbol{\alpha}_3=
		\begin{bmatrix}
		-1\\-2\\2
		\end{bmatrix}.
		$$
		将$\boldsymbol{\alpha}_1,\boldsymbol{\alpha}_2,\boldsymbol{\alpha}_3$正交化\\
		$$
		\begin{aligned}
		\boldsymbol{\beta}_1&=\boldsymbol{\alpha}_1=
		\begin{bmatrix}
		2\\0\\1
		\end{bmatrix}\\
		\boldsymbol{\beta}_2&=\boldsymbol{\alpha}_2-\frac{(\boldsymbol{\alpha}_2,\boldsymbol{\beta}_1)}{(\boldsymbol{\beta}_1,\boldsymbol{\beta}_1)}=
		\begin{bmatrix}
		-\frac{4}{5}\\0\\\frac{8}{5}
		\end{bmatrix}\\
		\boldsymbol{\beta}_3&=\boldsymbol{\alpha}_3-\frac{(\boldsymbol{\alpha}_3,\boldsymbol{\beta}_2)}{(\boldsymbol{\beta}_2,\boldsymbol{\beta}_2)}-\frac{(\boldsymbol{\alpha}_3,\boldsymbol{\beta}_1)}{(\boldsymbol{\beta}_1,\boldsymbol{\beta}_1)}=
		\begin{bmatrix}
		0\\-2\\1
		\end{bmatrix}
		\end{aligned}
		$$
		再将$\boldsymbol{\beta}_1,\boldsymbol{\beta}_2,\boldsymbol{\beta}_3$单位化得$\boldsymbol{\mu}_1,\boldsymbol{\mu}_2,\boldsymbol{\mu}_3$
		即
		$$
		\boldsymbol{T}=
		\begin{bmatrix}
		\frac{2}{\sqrt{5}}&-\frac{\sqrt{5}}{5}&0\\
		0&0&1\\
		\frac{1}{\sqrt{5}}&\frac{2\sqrt{5}}{5}&0
		\end{bmatrix}
		$$
		则$\boldsymbol{T}$是正交矩阵,且$\boldsymbol{T}^{-1}\boldsymbol{A}\boldsymbol{T}=diag\{2,2,-7\}$.\\
		令
		$$
		\begin{bmatrix}
		x_1\\x_2\\x_3
		\end{bmatrix}
		=
		\boldsymbol{T}
		\begin{bmatrix}
		y_1\\y_2\\y_3
		\end{bmatrix}.
		$$
		则\\
		$$f(y_1,y_2,y_3)=2y_1^2+2y_2^2-7y_3^2.$$
	\end{proof}
	\begin{exercise}
		设$f(x_1,x_2,x_3)=2x_1^2-x_2^2+ax_3^2=2x_1x_2-8x_1x_3+2x_2x_3$在正交变换$x=Qy$下的标准型为$\lambda_1y_1^2+\lambda_2y_2^2$,求$a$的值及一个正交矩阵$Q$.
	\end{exercise}
	\begin{proof}
		因为标准型为$\lambda_1y_1^2+\lambda_2y_2^2$所以$det(\boldsymbol{A})=0$得$a=2$:
		$$
		\boldsymbol{A}=
		\begin{bmatrix}
		2&1&-4\\
		1&-1&1\\
		-4&1&2
		\end{bmatrix}
		|\lambda\boldsymbol{I}-\boldsymbol{A}|=(\lambda-6)(\lambda+3)\lambda=0
		$$
		$\lambda=6,\lambda=-3,\lambda=0$的特征向量分别为:
		$$\boldsymbol{\alpha}_1=
		\begin{bmatrix}
		-1\\0\\1
		\end{bmatrix}
		\boldsymbol{\alpha}_2=
		\begin{bmatrix}
		1\\-1\\1
		\end{bmatrix}
		\boldsymbol{\alpha}_3=
		\begin{bmatrix}
		1\\2\\1
		\end{bmatrix}
		$$
		将$\boldsymbol{\alpha}_1,\boldsymbol{\alpha}_2,\boldsymbol{\alpha}_3$单位化,即得:
		$$
		\boldsymbol{T}=
		\begin{bmatrix}
		-\frac{1}{\sqrt{2}}&\frac{1}{\sqrt{3}}&\frac{1}{\sqrt{6}}\\
		0&-\frac{1}{\sqrt{3}}&\frac{2}{\sqrt{6}}\\
		\frac{1}{\sqrt{2}}&\frac{1}{\sqrt{3}}&\frac{1}{\sqrt{6}}
		\end{bmatrix}
		$$
		则$\boldsymbol{T}$是正交矩阵,且$\boldsymbol{T}^{-1}\boldsymbol{A}\boldsymbol{T}=diag\{6,-3,0\}$.\\
		令
		$$
		\begin{bmatrix}
		x_1\\x_2\\x_3
		\end{bmatrix}
		=
		\boldsymbol{T}
		\begin{bmatrix}
		y_1\\y_2\\y_3
		\end{bmatrix}.
		$$
		则\\
		$$f(y_1,y_2,y_3)=6y_1^2-3y_2^2.$$
	\end{proof}
	\begin{exercise}
		用正交变换化二次型
		$$
		f(x_1,\cdots,x_4)=2(x_1^2+x_2^2+x_3^2+x_4^2+x_1x_2+x_3x_4)
		$$
		为标准型,并给出所用的正交线性变化.
	\end{exercise}
	\begin{proof}
		二次型方程的矩阵$\boldsymbol{A}$为
		$$
		\boldsymbol{A}=
		\begin{bmatrix}
		2&1&0&0\\
		1&2&0&0\\
		0&0&2&1\\
		0&0&1&2
		\end{bmatrix},
		|\lambda\boldsymbol{I}-\boldsymbol{A}|=
		\begin{vmatrix}
		\lambda-2&-1&0&0\\
		-1&\lambda-2&0&0\\
		0&0&\lambda-2&-1\\
		0&0&-1&\lambda-2
		\end{vmatrix}
		=(\lambda-3)^2(\lambda-1)^2.
		$$
		$\boldsymbol{A}$的全部特征值是3,1.\\
		对特征值3,求出$(3\boldsymbol{I}-\boldsymbol{A})\boldsymbol{X}=0$的基础解系:
		$$
		\boldsymbol{\alpha}_1=
		\begin{bmatrix}
		0\\0\\1\\1
		\end{bmatrix},
		\boldsymbol{\alpha}_2=
		\begin{bmatrix}
		1\\1\\0\\0
		\end{bmatrix}.
		$$
		把$\boldsymbol{\alpha}_1,\boldsymbol{\alpha}_2$单位化,得\\
		对特征值1,求出$(\boldsymbol{I}-\boldsymbol{A})\boldsymbol{X}=0$的基础解系:
		$$
		\boldsymbol{\alpha}_3=
		\begin{bmatrix}
		0\\0\\-1\\1
		\end{bmatrix},
		\boldsymbol{\alpha}_4=
		\begin{bmatrix}
		-1\\1\\0\\0
		\end{bmatrix}.
		$$
		因$\boldsymbol{\alpha}_1,\boldsymbol{\alpha}_2,\boldsymbol{\alpha}_3,\boldsymbol{\alpha}_4$已经为正交向量组,即正交化得:
		$$
		\boldsymbol{T}=
		\begin{bmatrix}
		0&\frac{1}{\sqrt{2}}&0&-\frac{1}{\sqrt{2}}\\
		0&\frac{1}{\sqrt{2}}&0&\frac{1}{\sqrt{2}}\\
		\frac{1}{\sqrt{2}}&0&-\frac{1}{\sqrt{2}}&0\\
		\frac{1}{\sqrt{2}}&0&\frac{1}{\sqrt{2}}&0
		\end{bmatrix}
		\qquad\boldsymbol{T}'\boldsymbol{AT}=\boldsymbol{D}=diag(3,3,1,1);
		$$
		令
		$$
		\begin{bmatrix}
		x_1\\x_2\\x_3\\x_4
		\end{bmatrix}
		=
		\boldsymbol{T}
		\begin{bmatrix}
		y_1\\y_2\\y_3\\y_4
		\end{bmatrix}.
		$$
		则
		$$f(y_1,y_2,y_3,y_4)=3y_1^2+3y_2^2+y_3^2+y_4^2.$$
	\end{proof}
	\begin{exercise}
		求一个正交变换将二次型
		$$
		f=2x_1^2+3x_2^2+3x_3^2+4x_2x_3
		$$
		化成标准形.
	\end{exercise}
	\begin{proof}
		二次型矩阵$\boldsymbol{A}$为
		$$
		\boldsymbol{A}=
		\begin{bmatrix}
		2&0&0\\
		0&3&2\\
		0&2&3
		\end{bmatrix}
		$$
		解得$\lambda_1=5,\lambda_2=2,\lambda_3=1$以及对应的特征向量$\boldsymbol{\alpha}_1,\boldsymbol{\alpha}_2,\boldsymbol{\alpha}_3$为:
		$$
		\boldsymbol{\alpha}_1=
		\begin{bmatrix}
		0\\1\\1
		\end{bmatrix},
		\boldsymbol{\alpha}_2=
		\begin{bmatrix}
		1\\0\\0
		\end{bmatrix},
		\boldsymbol{\alpha}_3=
		\begin{bmatrix}
		0\\-1\\1
		\end{bmatrix}.
		$$
		对$\boldsymbol{\alpha}_1,\boldsymbol{\alpha}_2,\boldsymbol{\alpha}_3$单位化得:
		$$
		\boldsymbol{T}=
		\begin{bmatrix}
		0&1&0\\
		\frac{1}{\sqrt{2}}&0&-\frac{1}{\sqrt{2}}\\
		\frac{1}{\sqrt{2}}&0&\frac{1}{\sqrt{2}}
		\end{bmatrix}
		\qquad\boldsymbol{T}'\boldsymbol{AT}=\boldsymbol{D}=diag(5,2,1);
		$$
		令
		$$
		\begin{bmatrix}
		x_1\\x_2\\x_3
		\end{bmatrix}
		=
		\boldsymbol{T}
		\begin{bmatrix}
		y_1\\y_2\\y_3
		\end{bmatrix}.
		$$
		则
		$$f(y_1,y_2,y_3)=5y_1^2+2y_2^2+y_3^2.$$
	\end{proof}
	\begin{exercise}
		设$\boldsymbol{A}$是一个$n$阶正定矩阵,则$|\boldsymbol{A}+\boldsymbol{E}|>1$。
	\end{exercise}
	\begin{proof}
		设$\boldsymbol{A}$的n个实特征值为$\lambda_1,\cdots,\lambda_n$
		\\有正交矩阵$\boldsymbol{C}$使得$\boldsymbol{A}=(\boldsymbol{C}')^{-1}\boldsymbol{D}\boldsymbol{C}^{-1}$
		$$
		|\boldsymbol{A}+\boldsymbol{E}|=|\boldsymbol{CDC}'+\boldsymbol{E}|=|\boldsymbol{CDC}'+\boldsymbol{CC}'|=|\boldsymbol{C}||\boldsymbol{DC}'+\boldsymbol{C}'|=|\boldsymbol{D}+\boldsymbol{E}||\boldsymbol{C}|^2=(\lambda_1+1)\cdots(\lambda_n+1)>1
		$$
		且$\lambda_1,\cdots,\lambda_n>0$即
		$$
		|\boldsymbol{A}+\boldsymbol{E}|>1.
		$$
	\end{proof}
	\begin{exercise}
		设$n$阶实对称矩阵$\boldsymbol{A}$正定,证明$\boldsymbol{A}^*$也是正定矩阵,其中$\boldsymbol{A}^*$表示$\boldsymbol{A}$的伴随矩阵.
	\end{exercise}
	\begin{proof}
		存在实矩阵$\boldsymbol{C}$,使得
		$$\boldsymbol{A}=\boldsymbol{C}'\boldsymbol{C},\qquad\boldsymbol{A}^{-1}=\boldsymbol{C}^{-1}\boldsymbol{I}(\boldsymbol{C}^{-1})'
		$$
		即$\boldsymbol{A}^{-1}\backsimeq\boldsymbol{I}$得$\boldsymbol{A}^{-1}$是正定的.\\
		$$
		\begin{aligned}
		\boldsymbol{A}^*&=|\boldsymbol{A}|\boldsymbol{A}^{-1}\\
		&=|\boldsymbol{A}|\boldsymbol{C}^{-1}\boldsymbol{I}(\boldsymbol{C}^{-1})'\\
		&=(\sqrt{|\boldsymbol{A}|}\boldsymbol{C}^{-1})'\boldsymbol{I}(\sqrt{|\boldsymbol{A}|}\boldsymbol{C}^{-1})'
		\end{aligned}
		$$
		即$\boldsymbol{A}\backsimeq\boldsymbol{I}$
	\end{proof}
	\begin{exercise}
		设$n$阶方阵$\boldsymbol{A}$满足条件$\boldsymbol{A}'\boldsymbol{A}=\boldsymbol{E}$,其中$\boldsymbol{A}'$是$\boldsymbol{A}$的转置矩阵,$\boldsymbol{E}$为单位矩阵,证明$\boldsymbol{A}$的实特征向量所对应的特征值的绝对值等于1.
	\end{exercise}
	\begin{proof}
		已知$\boldsymbol{A}'=\boldsymbol{A}^{-1}$\\
		由于$|\lambda\boldsymbol{I}|=|(\lambda\boldsymbol{I}-\boldsymbol{A})'|=|\lambda\boldsymbol{I}-\boldsymbol{A}'|=|\lambda\boldsymbol{I}-\boldsymbol{A}^{-1}|=|\boldsymbol{A}^{-1}||\lambda\boldsymbol{I}-\boldsymbol{A}|=$
		$$
		\frac{1}{|\boldsymbol{A}|}\frac{1}{\lambda}(-1)^n|\lambda\boldsymbol{I}-\boldsymbol{A}|
		$$
		所以$\frac{1}{|\boldsymbol{A}|}\frac{1}{\lambda}(-1)^n=1$,而$|\boldsymbol{AA}'|=|\boldsymbol{A}|^2=1,$所以$|\lambda|=1$
	\end{proof}
	\begin{exercise}
		设$f(x_1,\cdots,x_n)=\boldsymbol{X}^T\boldsymbol{AX}$是一实二次型,若存在n维实向量
		$$
		\boldsymbol{X}_1,\boldsymbol{X}_2,s.t.,\boldsymbol{X}_1^T\boldsymbol{AX}_1>0,\boldsymbol{X}_2^T\boldsymbol{AX}_2<0,
		$$
		证明:存在$n$维实向量$\boldsymbol{a},s.t.,\boldsymbol{a}'\boldsymbol{A}\boldsymbol{a}=0.$
	\end{exercise}
	\begin{proof}
		设$\boldsymbol{A}$的秩为$r$,作实非退化线性替换$\boldsymbol{X}=\boldsymbol{CY}$,将$f$化为规范形
		$$
		f(x_1,\cdots,x_n)=\boldsymbol{X}'\boldsymbol{AX}=y_1^2+\cdots+y_p^2-y^2_{p+1}-\cdots-y^2_{p+q}(r=p+q)
		$$
		由于存在两个向量$\boldsymbol{X}_1,\boldsymbol{X}_2$,使$\boldsymbol{X}'_1\boldsymbol{AX}_1>0,\boldsymbol{X}'_2\boldsymbol{AX}_2<0$,,从而可得$p>0,q>0$.\\
		令
		$$
		y_1=y_{p+1}=1,y_2=\cdots=y_p=y_{p+2}=\cdots=y_{p+q}=0,
		$$
		取$n$维列向量
		$$
		\boldsymbol{X}_0=\boldsymbol{C}
		\begin{bmatrix}
		1\\0\\\vdots\\0\\1\\0\\\vdots\\0
		\end{bmatrix}
		$$
		中间的1位于第$p+1$行.\\
		由于$\boldsymbol{X}=\boldsymbol{CY}$非退化,故$\boldsymbol{X}_0\ne\boldsymbol{0}$,且有
		$$
		\boldsymbol{X}'_0\boldsymbol{AX}_0=0
		$$
	\end{proof}
	\begin{exercise}
		设$\boldsymbol{A}$是$n$阶实对称矩阵,证明:
		$$
		\exists c>0,s.t., \forall \boldsymbol{X}\in \boldsymbol{R}^n,|\boldsymbol{X}^T\boldsymbol{A}\boldsymbol{X}|\leq c\boldsymbol{X}^T\boldsymbol{X}.
		$$
	\end{exercise}
	\begin{proof}
		设$\boldsymbol{A}$的全部特征值是$\lambda_1,\cdots,\lambda_n$.则$t\boldsymbol{I}+\boldsymbol{A}$的全部特征值为$t+\lambda_1,\cdots,t+\lambda_n$,当$t>S_r(\boldsymbol{A})$时.
		$$
		t+\lambda_i\geq t-|\lambda_i|\geq t-S_r(\boldsymbol{A})>0,\;i=1,\cdots,n.
		$$
		即$t\boldsymbol{I}+\boldsymbol{A}$为正定矩阵.同理$t\boldsymbol{I}-\boldsymbol{A}$也为正交矩阵.\\
		任意一个实n阶向量$\boldsymbol{X}$,都有.
		$$
		\boldsymbol{X}'(c\boldsymbol{I}+\boldsymbol{A})\boldsymbol{X}\geqslant 0,\;\boldsymbol{X}'(c\boldsymbol{I}-\boldsymbol{A})\geqslant 0
		$$
		$$
		-c\boldsymbol{X}'\boldsymbol{X}\leqslant \boldsymbol{X}'\boldsymbol{AX}\leqslant c\boldsymbol{X}'\boldsymbol{X}
		$$
		即$$|\boldsymbol{X}'\boldsymbol{AX}|\leqslant c\boldsymbol{X}'\boldsymbol{X}$$
	\end{proof}
	\begin{exercise}
		(1)证明:如果${\displaystyle \sum^n_{i=1}\sum^n_{j=1}}a_{ij}x_ix_j(a_{ij}=a_{ji})$是正定二次型.那么
		$$
		f(y_1,\dots,y_n)=
		\begin{bmatrix}
		a_{11}&a_{12}&\cdots&a_{1n}&y_1\\
		a_{21}&a_{22}&\cdots&a_{2n}&y_2\\
		\vdots&\vdots&\ddots&\vdots&\vdots\\
		a_{n1}&a_{n2}&\cdots&a_{nn}&y_n\\
		y_1&y_2&\cdots&y_n&0
		\end{bmatrix}
		$$
		是负定二次型;\\
		(2)如果$\boldsymbol{A}=
		\begin{bmatrix}
		a_{11}&a_{12}&\cdots&a_{1n}\\
		a_{21}&a_{22}&\cdots&a_{2n}\\
		\vdots&\vdots&\ddots&\vdots\\
		a_{n1}&a_{n2}&\cdots&a_{nn}
		\end{bmatrix}
		$是正定矩阵,那么
		$$
		|\boldsymbol{A}|\leq a_{nn}\boldsymbol{P}_{n-1},
		$$
		其中$\boldsymbol{P}_{n-1}$是$\boldsymbol{A}$的n-1阶顺序主子式;\\
		(3)$|\boldsymbol{A}|\leq a_{11}a_{22}\cdots a_{nn}.$
	\end{exercise}
		\begin{proof}
		(1)作变换$\boldsymbol{Y}=\boldsymbol{AZ}$即
		$$
		\begin{bmatrix}
		y_1\\\vdots\\y_n
		\end{bmatrix}
		=
		\begin{bmatrix}
		a_{11}&\cdots&a_{1n}\\
		\vdots&      &\vdots\\
		a_{n1}&\cdots&a_{nn}
		\end{bmatrix}
		\begin{bmatrix}
		z_1\\\vdots\\z_n
		\end{bmatrix}
		$$
		则
		$$
		\begin{aligned}
		f(y_1,\cdots,y_n)&=
		\begin{vmatrix}
		a_{11}&\cdots&a_{1n}&a_{11}z_1+\cdots+a_{1n}z_n\\
		\vdots&&\vdots&\vdots\\
		a_{n1}&\cdots&a_{nn}&a_{n1}z_1+\cdots+a_{nn}z_n\\
		y_1&\cdots&y_n&0
		\end{vmatrix}\\
		&\xRightarrow[(i=1,\cdots,n-1)]{c_n-z_nc_i}
		\begin{vmatrix}
		a_{11}&\cdots&a_{1n}&0\\
		\vdots&&\vdots&\vdots\\
		a_{n1}&\cdots&a_{nn}&0\\
		y_1&\cdots&y_n&-(y_1z_1+\cdots y_nz_n)
		\end{vmatrix}\\
		&=-|\boldsymbol{A}|(y_1z_1+\cdots y_nz_n)\\
		&=-|\boldsymbol{A}|\boldsymbol{Y}'\boldsymbol{Z}=-|\boldsymbol{A}|\boldsymbol{Z}'\boldsymbol{AZ}
		\end{aligned}
		$$
		由$\boldsymbol{A}$为正定矩阵知,$f(y_1,y_2,\cdots,y_{n-1})$为负定二次型.\\
	\end{proof}
	\begin{proof}
		(2)记
		$$
		g'(y_1,y_2,\cdots,y_n)=
		\begin{vmatrix}
		a_{11}&a_{12}&\cdots&a_{1,n-1}&y_1\\
		a_{21}&a_{22}&\cdots&a_{2,n-1}&y_2\\
		\vdots&\vdots&&\vdots&\vdots\\
		a_{n1}&a_{n2}&\cdots&a_{n,n-1}&y_n\\
		y_1&y_2&\cdots&y_{n-1}&0
		\end{vmatrix}
		$$
		由(1)可知,$g'(y_1,y_2,\cdots,y_{n-1})$负定
		$$
		|\boldsymbol{A}|=g'(a_{n1},\cdots,a_{n,n-1})+a_{nn}\boldsymbol{P}_{n-1}
		$$
		由于$g'(a_{n1},\cdots,a_{n,n-1})<0$,所以$|\boldsymbol{A}|\leq a_{nn}\boldsymbol{\cdot P}_{n-1}$.
	\end{proof}
	\begin{proof}
		(3)由(2),得
		$$
		|\boldsymbol{A}|\leq a_{nn},|\boldsymbol{P}_{n-1}|\leq a_{nn}a_{n-1,n-1},|\boldsymbol{P}_{n-2}|\leq \cdots\leq a_{nn}a_{n-1,n-1}\cdots a_{11}
		$$
	\end{proof}
	\begin{exercise}
		设$\boldsymbol{A},\boldsymbol{B}$都是$n\times n$实对称矩阵,且$\boldsymbol{A}$正定,\\
		(1)证明:存在实可逆矩阵$\boldsymbol{T}$,使得$\boldsymbol{T}'(\boldsymbol{A}+\boldsymbol{B})\boldsymbol{T}$为对角矩阵;\\
		(2)假设$\boldsymbol{B}$也正定.证明:$|\boldsymbol{A}+\boldsymbol{B}|\geq|\boldsymbol{A}|+|\boldsymbol{B}|.$\\
		(3)假设$\boldsymbol{B}$也正定,且$\boldsymbol{A}\boldsymbol{B}=\boldsymbol{B}\boldsymbol{A}$,证明$\boldsymbol{A}\boldsymbol{B}$也是正定矩阵.
	\end{exercise}
	\begin{proof}
		(1)\\
		$\boldsymbol{A}$正定,所以存在正交矩阵$\boldsymbol{P}^{-1}=\boldsymbol{P'}$使得,
		$$
		\boldsymbol{P}'\boldsymbol{AP}=diag(a_1,\cdots,a_n),
		$$
		$a_1,\cdots,a_n>0$为$\boldsymbol{A}$的特征值.\\
		$\boldsymbol{P}'\boldsymbol{BP}=\boldsymbol{P}^{-1}\boldsymbol{BP}$也是对称矩阵,且与$\boldsymbol{B}$有相同的特征值.所以存在正交矩阵$\boldsymbol{Q}^{-1}=\boldsymbol{Q}'$使得,
		$$
		\boldsymbol{Q}'(\boldsymbol{P}'\boldsymbol{BP})\boldsymbol{Q}=\boldsymbol{Q^{-1}}(\boldsymbol{P}^{-1}\boldsymbol{BP})\boldsymbol{Q}=diag(b_1,\cdots,b_n),
		$$
		其中$b_1,\cdots,b_n$为$\boldsymbol{B}$的特征值.所以:
		$$
		\boldsymbol{Q}'(\boldsymbol{P}'(\boldsymbol{A}+\boldsymbol{B})\boldsymbol{P})\boldsymbol{Q}=\boldsymbol{Q}^{-1}(\boldsymbol{P}^{-1}(\boldsymbol{A}+\boldsymbol{B})\boldsymbol{P})\boldsymbol{Q}=diag(a_1+b_1,\cdots,a_n+b_n).
		$$
		所以存在实可逆矩阵$\boldsymbol{T}=\boldsymbol{PQ}$,使得$\boldsymbol{T}'(\boldsymbol{A}+\boldsymbol{B})\boldsymbol{T}$为对角矩阵.
	\end{proof}
	\begin{proof}
		(2)\\
		$\boldsymbol{B}$也正定,因此$b_1,\cdots,b_n>0$.
		\begin{align*}
			|\boldsymbol{Q}'(\boldsymbol{P}'(\boldsymbol{A}+\boldsymbol{B})\boldsymbol{P})\boldsymbol{Q}|
			&=|diag(a_1+b_1,\cdots,a_n+b_n)|\\
			&=\prod_{i=1}^{n}(a_i+b_i)\\
			&>\prod_{i=1}^{n}a_i+\prod_{i=1}^{n}b_i\\
			&=|\boldsymbol{P}^{-1}\boldsymbol{AP}|+|\boldsymbol{P}^{-1}\boldsymbol{BP}|=|\boldsymbol{A}|+|\boldsymbol{B}|.
			\end{align*}
			$$
			\Rightarrow|\boldsymbol{A}+\boldsymbol{B}|\ge|\boldsymbol{A}|+|\boldsymbol{B}|.
			$$
	\end{proof}
	\begin{proof}
		(3)存在一个n级正交矩阵$\boldsymbol{T},s.t.$
		$$
		\boldsymbol{T}'\boldsymbol{AT}=diag\{\lambda_1,\cdots,\lambda_n\}\quad\boldsymbol{T}'\boldsymbol{BT}=diag\{\mu_1,\cdots,\mu_n\}.
		$$
		其中$\lambda_1,\cdots,\lambda_n$是$\boldsymbol{A}$的全部特征值,$\mu_1,\cdots,\mu_n$是$\boldsymbol{B}$的全部特征值.$\boldsymbol{A},\boldsymbol{B}$都是正定矩阵,由于:
		$$
		\boldsymbol{T}'(\boldsymbol{AB})\boldsymbol{T}=(\boldsymbol{T}'\boldsymbol{AT})(\boldsymbol{T}'\boldsymbol{BT})=diag\{\lambda_1\mu_1,\cdots,\lambda_n\mu_n \}
		$$
		即$\boldsymbol{AB}$合同与$diag\{\lambda_1\mu_1,\cdots,\lambda_n\mu_n \}$,所以$\boldsymbol{AB}$是正定矩阵.
	\end{proof}
	\begin{exercise}
		化二次型
		$$
		f(x_1,x_2,x_3)=2x_1^2+4x_1x_2-4x_1x_3+5x_2^2-8x_2x_3+5x_3^2
		$$
		为标准型并写出线性变换.
	\end{exercise}
	\begin{proof}
		二次型f的矩阵$\boldsymbol{A}$,以及$\boldsymbol{A}$的全部特征值以及特征向量为:
		$$
		\boldsymbol{A}=
		\begin{bmatrix}
		2&2&-2\\
		2&5&-4\\
		-2&-4&5
		\end{bmatrix};
		$$
		$\lambda_1,\lambda_2,\lambda_3$的特征值分别为10,1,1特征向量分别为$$\boldsymbol{v}_1=\begin{bmatrix}-1\\-2\\2\end{bmatrix}\;\boldsymbol{v}_2=\begin{bmatrix}2\\0\\1\end{bmatrix}\;\boldsymbol{v}_3=\begin{bmatrix}-2\\1\\0\end{bmatrix}$$
		将$\boldsymbol{v}_1,\boldsymbol{v}_2,\boldsymbol{v}_3$正交化,单位化得$\boldsymbol{\beta}_1,\boldsymbol{\beta}_2,\boldsymbol{\beta}_3$:
		$$
		\boldsymbol{\beta}_1=
		\begin{bmatrix}
		-\frac{1}{3}\\-\frac{2}{3}\\\frac{2}{3}
		\end{bmatrix}
		\boldsymbol{\beta}_2=
		\begin{bmatrix}
		\frac{2}{\sqrt{5}}\\0\\\frac{1}{\sqrt{5}}
		\end{bmatrix}
		\boldsymbol{\beta}_3=
		\begin{bmatrix}
		-\frac{2}{3\sqrt{5}}\\\frac{5}{3\sqrt{5}}\\\frac{4}{3\sqrt{5}}
		\end{bmatrix}
		$$
		即
		$$
		\boldsymbol{T}=
		\begin{bmatrix}
		-\frac{1}{3}&\frac{2}{\sqrt{5}}&-\frac{2}{3\sqrt{5}}\\
		-\frac{2}{3}&0&\frac{5}{3\sqrt{5}}\\
		\frac{2}{3}&\frac{1}{\sqrt{5}}&\frac{4}{3\sqrt{5}}
		\end{bmatrix}
		\qquad\boldsymbol{T}'\boldsymbol{AT}=\boldsymbol{D}=diag(10,1,1);
		$$
		令
		$$
		\begin{bmatrix}
		x_1\\x_2\\x_3
		\end{bmatrix}
		=
		\boldsymbol{T}
		\begin{bmatrix}
		y_1\\y_2\\y_3\
		\end{bmatrix}.
		$$
		则
		$$f(y_1,y_2,y_3)=10y_1^2+y_2^2+y_3^2.$$
	\end{proof}
	\begin{exercise}
		设二次型
		$$
		f(x_1,x_2,x_3)=ax_1^2+ax_2^2+(a-1)x_3^2+2x_1x_3-2x_2x_3.
		$$
		(1)求二次型$f$的矩阵的所有特征值;\\
		(2)若二次型的规范性为$y_1^2+y_2^2$,求a的值和各特征值的一个特征向量.
	\end{exercise}
	\begin{proof}
		(1)二次型的矩阵$\boldsymbol{A}$为:
		$$
		\boldsymbol{A}=
		\begin{bmatrix}
		a&0&1\\
		0&a&-1\\
		1&-1&a-1
		\end{bmatrix}\;
		|\lambda\boldsymbol{I}-\boldsymbol{A}|=
		\begin{vmatrix}
		\lambda-a&0&-1\\
		0&\lambda-a&1\\
		-1&1&\lambda-(a-1)
		\end{vmatrix}
		$$
		解得$\lambda_1,\lambda_2,\lambda_3$的特征值分别为$a-2,a,a+1$,特征向量分别为$$\boldsymbol{v}_1=\begin{bmatrix}-1\\1\\2\end{bmatrix},\;\boldsymbol{v}_2=\begin{bmatrix}1\\1\\0\end{bmatrix},\;\boldsymbol{v}_3=\begin{bmatrix}1\\-1\\1\end{bmatrix}.$$
	\end{proof}
	\begin{proof}
		(2)若规范性为$y_1^2+y_2^2,a$只能为$2$,即:\\
		$\lambda_1,\lambda_2,\lambda_3$的特征值分别为$0,2,3$,特征向量分别为$$\boldsymbol{v}_1=\begin{bmatrix}-1\\1\\2\end{bmatrix},\;\boldsymbol{v}_2=\begin{bmatrix}1\\1\\0\end{bmatrix},\;\boldsymbol{v}_3=\begin{bmatrix}1\\-1\\1\end{bmatrix}.$$
	\end{proof}
	\begin{exercise}
		若$n\times n$矩阵$\boldsymbol{A}$可逆,\\
		(1)证明$\boldsymbol{A}'\boldsymbol{A}$正定.\\
		(2)证明:存在正交矩阵$\boldsymbol{P},\boldsymbol{Q}$,使得
		$$
		\boldsymbol{P}'\boldsymbol{AQ}=diag(a_1,\cdots,a_n),where \; a_i>0,i=1,\cdots,n.
		$$
	\end{exercise}
	\begin{proof}
		(1)
		$$
		\boldsymbol{A}'\boldsymbol{A}=\boldsymbol{A}'\boldsymbol{IA}
		$$
		因此$\boldsymbol{A}'\boldsymbol{A}$合同与单位矩阵,即$\boldsymbol{A}'\boldsymbol{A}$正定.
	\end{proof}
	\begin{proof}
		(2)
		$\boldsymbol{A}'\boldsymbol{A}$正定,因此存在正交矩阵$\boldsymbol{P}$,使得
		$$
		\boldsymbol{P}'\boldsymbol{A}'\boldsymbol{AP}=diag(b_1,\cdots,b_n),where\;b_i>0,i=0,\cdots,n.
		$$
		令
		$$
		\boldsymbol{D}=diag(\sqrt{b_1},\cdots,\sqrt{b_n}),\;\boldsymbol{Q}=\boldsymbol{A}'\boldsymbol{PD},
		$$
		容易求得
		$$
		\boldsymbol{Q}'\boldsymbol{Q}=(\boldsymbol{A}'\boldsymbol{PD})'\boldsymbol{A}'\boldsymbol{PD}=\boldsymbol{I}_n,
		$$
		因此矩阵$\boldsymbol{Q}$为正交矩阵,令$a_i=\sqrt{b_i},i=1\cdots,n.$则
		$$
		\boldsymbol{P}'\boldsymbol{AQ}=\boldsymbol{D}=diag(a_1,\cdots,a_n).
		$$
	\end{proof}
	\begin{exercise}
		求线性方程组
		$
		\begin{cases}
		2x_1+x_2+3_3x_3-x_4=0\\
		3x_1+2x_2+0x_3-2x_4=0\\
		3x_1+x_2+9x_3-x_4=0
		\end{cases}
		$的解空间$W$及其正交补空间$W^{\perp}$.
	\end{exercise}
	\begin{proof}
		系数矩阵作初等行变换得:
		$$
		\boldsymbol{A}=
		\begin{bmatrix}
		2&1&3&-1\\
		3&2&0&-2\\
		3&1&9&-1
		\end{bmatrix}
		\xRightarrow{}
		\begin{bmatrix}
		1&0&6&0\\
		0&1&-9&-1\\
		0&0&0&0
		\end{bmatrix}
		$$
		因此,方程组的两个线性无关的向量为:
		$$
		\boldsymbol{\alpha}_1=
		\begin{bmatrix}
		0\\1\\0\\1
		\end{bmatrix},\;
		\boldsymbol{\alpha}_2=
		\begin{bmatrix}
		-6\\9\\1\\0
		\end{bmatrix}.
		$$
		$W=L(\boldsymbol{\alpha}_1,\boldsymbol{\alpha}_2)$.
		设正交补空间$W^\perp$的任一向量为$\boldsymbol{\beta}=(y_1,y_2,y_3,y_4)'$.则$\boldsymbol{\alpha}_1'\boldsymbol{\beta}=\boldsymbol{\alpha}_2'\boldsymbol{\beta}=0$.即
		$$
		\begin{cases}
		y_2+y_4=0\\
		-6y_1+9y_2+y_3=0
		\end{cases}
		$$
		因此,上面的方程组的两个线性无关的向量为:
		$$
		\boldsymbol{\beta}_1=
		\begin{bmatrix}
		1\\0\\6\\0
		\end{bmatrix},\;
		\boldsymbol{\beta}_2=
		\begin{bmatrix}
		-3\\-2\\0\\2
		\end{bmatrix}.
		$$
		所以,正交补空间$W^\perp=L(\boldsymbol{\beta}_1,\boldsymbol{\beta}_2)$.
	\end{proof}
	\begin{exercise}
		设$\sigma$是有限维线性空间$V$的可逆线性变换.设$W$是$V$中$\sigma-$不变子空间,证明:$W$是$V$中$\sigma^{-1}-$不变子空间.
	\end{exercise}
	\begin{proof}
		已知$\boldsymbol{\alpha}\in W,\sigma(\boldsymbol{\alpha})\in W$.\\
		设$\sigma(\boldsymbol{\alpha})=\boldsymbol{\beta}$,因$\sigma$可逆,即若:
		$$
		\sigma^{-1}(\boldsymbol{\beta})=\boldsymbol{\alpha}\qquad\boldsymbol{\beta}\;and\;\sigma^{-1}(\boldsymbol{\beta})\in W
		$$
		所以$W$是$V$中$\sigma^{-1}-$不变子空间
	\end{proof}
	下证$\sigma$为双射:
	\begin{proof}
		设$\sigma$为可逆变换,它的逆变换为$\sigma^{-1}$.\\
		任取$\boldsymbol{\xi},\boldsymbol{\eta}\in W$,且$\boldsymbol{\xi}\ne\boldsymbol{\eta}$,则必有$\sigma(\boldsymbol{\xi})\ne\sigma(\boldsymbol{\eta})$.不然,设$\sigma(\boldsymbol{\xi})=\sigma(\boldsymbol{\eta})$,两边左乘$\sigma^{-1}$,有$\boldsymbol{\xi}=\boldsymbol{\eta}$,这与条件矛盾,所以$\sigma$是单射.\\
		其次,对任一向量$\boldsymbol{\xi}\in W$,必有$\boldsymbol{\eta}$使$\sigma(\boldsymbol{\xi})=\boldsymbol{\eta}$,事实上,令$\sigma^{-1}(\boldsymbol{\eta})=\boldsymbol{\xi}$即可,所以$\sigma$是满射,故$\sigma$是双射,即$\sigma^{-1}(\boldsymbol{\beta})=\boldsymbol{\alpha}$.
	\end{proof}

	\begin{exercise}
		设$\sigma$是正交变换,证明:$\sigma$的不变子空间的正交补也是$\sigma$的不变子空间.
	\end{exercise}
	\begin{proof}
		设$W$是$\sigma$的任意一个不变子空间,取$W$的一组标准正交基$\epsilon_1,\cdots,\epsilon_n$,把它扩成$V$的一组标准正交基$\epsilon_1,\cdots,\epsilon_m,\epsilon_{m+1},\cdots,\epsilon_n$,由
		$$
		dim(W)+dim(W^\perp)=n
		$$
		$$
		W=L(\epsilon_1,\cdots,\epsilon_m),\;W^\perp=L(\epsilon_{m+1},\cdots,\epsilon_n).
		$$
		又$\sigma$是正交变换,则$\sigma(\epsilon_1),\cdots,\sigma(\epsilon_n)$也是标准正交基,由于$W$是$\sigma$的不变子空间,所以$\sigma(\epsilon_1),\cdots,\sigma(\epsilon_m)\in W$.且为$W$的一组标准正交基,从而$\sigma(\epsilon_{m+1},\cdots,\sigma(\epsilon_n))\in W^\perp$\\
		任取$\alpha\in W^\perp$则
		$$
		\alpha=k_{m+1}\epsilon_{m+1}+\cdots+k_n\epsilon_n\in W^\perp,
		$$
		所以
		$$
		\sigma(\alpha)=k_{m+1}\epsilon_{m+1}+\cdots+k_n\sigma(\epsilon_n)\in W^\perp.
		$$
		$W^\perp$是$\sigma$的不变子空间.
	\end{proof}
	\begin{exercise}
		证明:反对称矩阵的特征值为0或者纯虚数。
	\end{exercise}
	\begin{proof}
		设$\boldsymbol{A}$是反对称实矩阵,$\lambda$是$\boldsymbol{A}$的一个特征值,$\boldsymbol{\xi}$为相应的特征向量,即$\boldsymbol{A\xi}=\lambda\boldsymbol{\xi}$,则
		$$
		\bar{\boldsymbol{\xi}}'\boldsymbol{A\xi}=\bar{\boldsymbol{\xi}}'(-\boldsymbol{A}')\boldsymbol{\xi}=-\bar{\boldsymbol{\xi}}'\boldsymbol{A}'\boldsymbol{\xi}=-(\boldsymbol{A}\bar{\boldsymbol{\xi}})'\boldsymbol{\xi}=-(\overline{\boldsymbol{A\xi}})'\boldsymbol{\xi}
		$$
		即有$\lambda\bar{\boldsymbol{\xi}}\boldsymbol{\xi}=-\bar{\lambda}\bar{\boldsymbol{\xi}}\boldsymbol{\xi}$,从而$\lambda=-\bar{\lambda}$.\\
		令$\lambda=a+ib$,代入上式得$a+ib=-(a-ib)$,即有$a=0$.故$\lambda$是0或纯虚数.
	\end{proof}
	\begin{exercise}
		如果$\lambda$是正交矩阵$\boldsymbol{A}$的特征值,则$\frac{1}{\lambda}$也是正交矩阵$\boldsymbol{A}$的特征值.
	\end{exercise}
	\begin{proof}
		设$\lambda$是$\boldsymbol{A}$的特征值,已知$\lambda^{-1}$是$\boldsymbol{A}^{-1}$的特征值,由于$\boldsymbol{A}$是正交矩阵,$\boldsymbol{A}'=\boldsymbol{A}^{-1}$,所以$\lambda^{-1}$是$\boldsymbol{A}'$的特征值.因为$\boldsymbol{A}$与$\boldsymbol{A}'$有相同的特征值,所以$\lambda^{-1}$也是$\boldsymbol{A}$的特征值.
	\end{proof}
	\begin{exercise}
		设$\boldsymbol{A},\boldsymbol{B}$都是实对称矩阵,证明:存在正交矩阵$\boldsymbol{Q},s.t.,\boldsymbol{Q}^{-1}\boldsymbol{AQ}=\boldsymbol{B}$的充要条件为$\boldsymbol{A},\boldsymbol{B}$的特征多项式的根全部相同.
	\end{exercise}
	\begin{proof}
		必要性.若$\boldsymbol{T}'\boldsymbol{AT}=\boldsymbol{B}$,即$\boldsymbol{A},\boldsymbol{B}$相似,则$\boldsymbol{A},\boldsymbol{B}$的特征多项式相同,所以它们的特征根相同.\\
		充分性.设$\lambda_1,\lambda_2,\cdots,\lambda_n$为$\boldsymbol{A}$和$\boldsymbol{B}$的特征值,由$\boldsymbol{A},\boldsymbol{B}$为实对称矩阵,则存在正交矩阵$\boldsymbol{X}$和$\boldsymbol{Y}$,使
		$$
		\boldsymbol{X}^{-1}\boldsymbol{AX}=
		\begin{bmatrix}
		\lambda_1&&\\
		&\ddots&\\
		&&\lambda_n
		\end{bmatrix}
		=\boldsymbol{Y}^{-1}\boldsymbol{BY}
		$$
		于是
		$$
		\boldsymbol{YX}^{-1}\boldsymbol{AXY}^{-1}=\boldsymbol{B}
		$$
		令$\boldsymbol{T}=\boldsymbol{XY}^{-1}$,由$\boldsymbol{X},\boldsymbol{Y}$是正交矩阵知,$\boldsymbol{T}$也是正交矩阵,且有$\boldsymbol{T}^{-1}\boldsymbol{AT}=\boldsymbol{B}$.
	\end{proof}
	\begin{exercise}
		设$\boldsymbol{A}$是$n$阶实对称矩阵,且$\boldsymbol{A}^2=\boldsymbol{A}$.证明:存在正交矩阵
		$$
		\boldsymbol{Q},s.t.,\boldsymbol{Q}^{-1}\boldsymbol{AQ}=diag(1,\cdots,1,0,\cdots,0).
		$$
	\end{exercise}
	\begin{proof}
		设$\lambda$是$\boldsymbol{A}$的任意特征值,$\boldsymbol{\xi}$是属于$\lambda$的特征向量,则$\boldsymbol{A\xi}=\lambda\boldsymbol{\xi}$,从而
		$$
		\boldsymbol{A}^2\boldsymbol{\xi}=\boldsymbol{A}(\boldsymbol{A\xi})=\boldsymbol{A}(\lambda\boldsymbol{\xi})=\lambda\boldsymbol{A\xi}=\lambda^2\boldsymbol{\xi}.
		$$
		又由于$\boldsymbol{A}^2=\boldsymbol{A}$,所以有$\lambda^2\boldsymbol{\xi}=\boldsymbol{A}^2\boldsymbol{\xi}=\boldsymbol{A\xi}=\lambda\boldsymbol{\xi}$,即$(\lambda^2-\lambda)\boldsymbol{\xi}=0$,因为$\boldsymbol{\xi}\ne\boldsymbol{0}$,故$\lambda^2-\lambda=0$,从而$\lambda=0$或1,再由定理7知,存在正交矩阵$\boldsymbol{Q}$,使
		$$
		\boldsymbol{Q}^{-1}\boldsymbol{AQ}=diag(1,\cdots,1,0,\cdots,0).
		$$
		在上式中的对角线元素中,1的个数为$\boldsymbol{A}$的特征值1的个数,0的个数为$\boldsymbol{A}$的特征值0的个数.
	\end{proof}
\end{document}
