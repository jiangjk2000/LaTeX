\documentclass[lang=cn,11pt,normal]{elegantbook}
\usepackage{mathtools}
\usepackage{LAbf}
% title info
\title{Linear Algebra}
\subtitle{高等代数作业}
% bio info
\author{JJK}
\institute{Jiang Xi science and technology University}
\date{\today}
% extra info
\version{1.00}
\extrainfo{Even though the truth is not a happy time, we have to be honest ,because more effort needed to cover the truth. --- Bertr Russell}
\logo{logo.png}
\cover{cover1.jpg}


\begin{document}
	\maketitle
	\mainmatter
	\hypersetup{pageanchor=true}
	% add preface chapter here if needed
	\chapter{高等代数}
	\section{作业}
	\begin{exercise}
		用正交变换化二次型$f(x_1,x_2,x_3)=x_1^2-2x_2^2-2x_3^2-4x_1x_2+4x_1x_3+8x_2x_3$为标准型,并给出所用的正交变换.
	\end{exercise}
	\begin{solution}
		这个实二次型的矩阵$\AA$为\\
		\begin{equation}
			\AA=
			\begin{bmatrix}
			1&-2&2\\
			-2&-2&4\\
			2&4&-2
			\end{bmatrix}.
		\end{equation}
		\begin{equation}
			|\lambda\II-\AA|=
			\begin{vmatrix}
			\lambda-1&2&-2\\
			2&\lambda+2&-4\\
			-2&2-4&\lambda+2
			\end{vmatrix}
			=(\lambda-2)^2(\lambda+7).
		\end{equation}
		$\AA$的全部特征值是2(二重),-7.\\
		对特征值2,求出$(2\II-\AA)\XX=0$的基础解系:
		\begin{equation}
		\bfaf_1=
		\begin{bmatrix}
		2\\
		0\\
		1
		\end{bmatrix},
		\bfaf_2=
		\begin{bmatrix}
		-2\\
		-1\\
		0
		\end{bmatrix}.
		\end{equation}
		对特征值-7,求出$(-7\II-\AA)\XX=0$的基础解系:\\
		\begin{equation}
		\bfaf_3=
		\begin{bmatrix}
		-1\\
		-2\\
		2
		\end{bmatrix}.
		\end{equation}
		得到三个线性无关的特征向量
		\begin{equation}
		\bfaf_1=
		\begin{bmatrix}
		2\\
		0\\
		1
		\end{bmatrix},
		\bfaf_2=
		\begin{bmatrix}
		-2\\
		-1\\
		0
		\end{bmatrix},
		\bfaf_3=
		\begin{bmatrix}
		-1\\
		-2\\
		2
		\end{bmatrix}.
		\end{equation}
		将$\bfaf_1,\bfaf_2,\bfaf_3$正交化\\
		\begin{equation}
		\begin{aligned}
		\bfbt_1&=\bfaf_1=
		\begin{bmatrix}
		2\\0\\1
		\end{bmatrix}\\
		\bfbt_2&=\bfaf_2-\frac{(\bfaf_2,\bfbt_1)}{(\bfbt_1,\bfbt_1)}=
		\begin{bmatrix}
		-\frac{4}{5}\\0\\\frac{8}{5}
		\end{bmatrix}\\
		\bfbt_3&=\bfaf_3-\frac{(\bfaf_3,\bfbt_2)}{(\bfbt_2,\bfbt_2)}-\frac{(\bfaf_3,\bfbt_1)}{(\bfbt_1,\bfbt_1)}=
		\begin{bmatrix}
		0\\-2\\1
		\end{bmatrix}
		\end{aligned}
		\end{equation}
		再将$\bfbt_1,\bfbt_2,\bfbt_3$单位化得$\boldsymbol{\mu}_1,\boldsymbol{\mu}_2,\boldsymbol{\mu}_3$
		即
		\begin{equation}
		\TT=
		\begin{bmatrix}
		\frac{2}{\sqrt{5}}&-\frac{\sqrt{5}}{5}&0\\
		0&0&1\\
		\frac{1}{\sqrt{5}}&\frac{2\sqrt{5}}{5}&0
		\end{bmatrix}
		\end{equation}
		则$\TT$是正交矩阵,且$\TT^{-1}\AA\TT=diag\{2,2,-7\}$.\\
		令
		\begin{equation}
		\begin{bmatrix}
		x_1\\x_2\\x_3
		\end{bmatrix}
		=
		\TT
		\begin{bmatrix}
		y_1\\y_2\\y_3
		\end{bmatrix}.
		\end{equation}
		则\\
		\begin{equation}
		f(y_1,y_2,y_3)=2y_1^2+2y_2^2-7y_3^2.
		\end{equation}
		\end{solution}
	\begin{exercise}
		设$f(x_1,x_2,x_3)=2x_1^2-x_2^2+ax_3^2=2x_1x_2-8x_1x_3+2x_2x_3$在正交变换$x=Qy$下的标准型为$\lambda_1y_1^2+\lambda_2y_2^2$,求$a$的值及一个正交矩阵$Q$.
	\end{exercise}
	\begin{solution}
		因为标准型为$\lambda_1y_1^2+\lambda_2y_2^2$所以$det(\AA)=0$得$a=2$:
		\begin{equation}
		\AA=
		\begin{bmatrix}
		2&1&-4\\
		1&-1&1\\
		-4&1&2
		\end{bmatrix}
		|\lambda\II-\AA|=(\lambda-6)(\lambda+3)\lambda=0
		\end{equation}
		$\lambda=6,\lambda=-3,\lambda=0$的特征向量分别为:
		\begin{equation}\bfaf_1=
		\begin{bmatrix}
		-1\\0\\1
		\end{bmatrix}
		\bfaf_2=
		\begin{bmatrix}
		1\\-1\\1
		\end{bmatrix}
		\bfaf_3=
		\begin{bmatrix}
		1\\2\\1
		\end{bmatrix}
		\end{equation}
		将$\bfaf_1,\bfaf_2,\bfaf_3$单位化,即得:
		\begin{equation}
		\TT=
		\begin{bmatrix}
		-\frac{1}{\sqrt{2}}&\frac{1}{\sqrt{3}}&\frac{1}{\sqrt{6}}\\
		0&-\frac{1}{\sqrt{3}}&\frac{2}{\sqrt{6}}\\
		\frac{1}{\sqrt{2}}&\frac{1}{\sqrt{3}}&\frac{1}{\sqrt{6}}
		\end{bmatrix}
		\end{equation}
		则$\TT$是正交矩阵,且$\TT^{-1}\AA\TT=diag\{6,-3,0\}$.\\
		令
		\begin{equation}
		\begin{bmatrix}
		x_1\\x_2\\x_3
		\end{bmatrix}
		=
		\TT
		\begin{bmatrix}
		y_1\\y_2\\y_3
		\end{bmatrix}.
		\end{equation}
		则\\
		\begin{equation}
		f(y_1,y_2,y_3)=6y_1^2-3y_2^2.
		\end{equation}
	\end{solution}
	\begin{exercise}
		用正交变换化二次型
		\begin{equation}
		f(x_1,\cdots,x_4)=2(x_1^2+x_2^2+x_3^2+x_4^2+x_1x_2+x_3x_4)
		\end{equation}
		为标准型,并给出所用的正交线性变化.
	\end{exercise}
	\begin{solution}
		二次型方程的矩阵$\AA$为
		\begin{equation}
		\AA=
		\begin{bmatrix}
		2&1&0&0\\
		1&2&0&0\\
		0&0&2&1\\
		0&0&1&2
		\end{bmatrix},
		|\lambda\II-\AA|=
		\begin{vmatrix}
		\lambda-2&-1&0&0\\
		-1&\lambda-2&0&0\\
		0&0&\lambda-2&-1\\
		0&0&-1&\lambda-2
		\end{vmatrix}
		=(\lambda-3)^2(\lambda-1)^2.
		\end{equation}
		$\AA$的全部特征值是3,1.\\
		对特征值3,求出$(3\II-\AA)\XX=0$的基础解系:
		\begin{equation}
		\bfaf_1=
		\begin{bmatrix}
		0\\0\\1\\1
		\end{bmatrix},
		\bfaf_2=
		\begin{bmatrix}
		1\\1\\0\\0
		\end{bmatrix}.
		\end{equation}
		把$\bfaf_1,\bfaf_2$单位化,得\\
		对特征值1,求出$(\II-\AA)\XX=0$的基础解系:
		\begin{equation}
		\bfaf_3=
		\begin{bmatrix}
		0\\0\\-1\\1
		\end{bmatrix},
		\bfaf_4=
		\begin{bmatrix}
		-1\\1\\0\\0
		\end{bmatrix}.
		\end{equation}
		因$\bfaf_1,\bfaf_2,\bfaf_3,\bfaf_4$已经为正交向量组,即正交化得:
		\begin{equation}
		\TT=
		\begin{bmatrix}
		0&\frac{1}{\sqrt{2}}&0&-\frac{1}{\sqrt{2}}\\
		0&\frac{1}{\sqrt{2}}&0&\frac{1}{\sqrt{2}}\\
		\frac{1}{\sqrt{2}}&0&-\frac{1}{\sqrt{2}}&0\\
		\frac{1}{\sqrt{2}}&0&\frac{1}{\sqrt{2}}&0
		\end{bmatrix}
		\qquad\TT'\AA\TT=\DD=diag(3,3,1,1);
		\end{equation}
		令
		\begin{equation}
		\begin{bmatrix}
		x_1\\x_2\\x_3\\x_4
		\end{bmatrix}
		=
		\TT
		\begin{bmatrix}
		y_1\\y_2\\y_3\\y_4
		\end{bmatrix}.
		\end{equation}
		则
		\begin{equation}
		f(y_1,y_2,y_3,y_4)=3y_1^2+3y_2^2+y_3^2+y_4^2.
		\end{equation}
	\end{solution}
	\begin{exercise}
		求一个正交变换将二次型
		\begin{equation}
		f=2x_1^2+3x_2^2+3x_3^2+4x_2x_3
		\end{equation}
		化成标准形.
	\end{exercise}
	\begin{solution}
		二次型矩阵$\AA$为
		\begin{equation}
		\AA=
		\begin{bmatrix}
		2&0&0\\
		0&3&2\\
		0&2&3
		\end{bmatrix}
		\end{equation}
		解得$\lambda_1=5,\lambda_2=2,\lambda_3=1$以及对应的特征向量$\bfaf_1,\bfaf_2,\bfaf_3$为:
		\begin{equation}
		\bfaf_1=
		\begin{bmatrix}
		0\\1\\1
		\end{bmatrix},
		\bfaf_2=
		\begin{bmatrix}
		1\\0\\0
		\end{bmatrix},
		\bfaf_3=
		\begin{bmatrix}
		0\\-1\\1
		\end{bmatrix}.
		\end{equation}
		对$\bfaf_1,\bfaf_2,\bfaf_3$单位化得:
		\begin{equation}
		\TT=
		\begin{bmatrix}
		0&1&0\\
		\frac{1}{\sqrt{2}}&0&-\frac{1}{\sqrt{2}}\\
		\frac{1}{\sqrt{2}}&0&\frac{1}{\sqrt{2}}
		\end{bmatrix}
		\qquad\TT'\AA\TT=\DD=diag(5,2,1);
		\end{equation}
		令
		\begin{equation}
		\begin{bmatrix}
		x_1\\x_2\\x_3
		\end{bmatrix}
		=
		\TT
		\begin{bmatrix}
		y_1\\y_2\\y_3
		\end{bmatrix}.
		\end{equation}
		则
		\begin{equation}
		f(y_1,y_2,y_3)=5y_1^2+2y_2^2+y_3^2.
		\end{equation}
	\end{solution}
	\begin{exercise}
		设$\AA$是一个$n$阶正定矩阵,则$|\AA+\II|>1$。
	\end{exercise}
	\begin{proof}
		设$\AA$的n个实特征值为$\lambda_1,\cdots,\lambda_n$
		\\有正交矩阵$\CC$使得$\AA=(\CC')^{-1}\DD\CC^{-1}$
		\begin{equation}
		|\AA+\II|=|\CC\DD\CC'+\II|=|\CC\DD\CC'+\CC\CC'|=|\CC||\DD\CC'+\CC'|=|\DD+\II||\CC|^2=(\lambda_1+1)\cdots(\lambda_n+1)>1
		\end{equation}
		且$\lambda_1,\cdots,\lambda_n>0$即
		\begin{equation}
		|\AA+\II|>1.
		\end{equation}
	\end{proof}
	\begin{exercise}
		设$n$阶实对称矩阵$\AA$正定,证明$\AA^*$也是正定矩阵,其中$\AA^*$表示$\AA$的伴随矩阵.
	\end{exercise}
	\begin{proof}
		存在实矩阵$\CC$,使得
		\begin{equation}\AA=\CC'\CC,\qquad\AA^{-1}=\CC^{-1}\II(\CC^{-1})'
		\end{equation}
		即$\AA^{-1}\backsimeq\II$得$\AA^{-1}$是正定的.\\
		\begin{equation}
		\begin{aligned}
		\AA^*&=|\AA|\AA^{-1}\\
		&=|\AA|\CC^{-1}\II(\CC^{-1})'\\
		&=(\sqrt{|\AA|}\CC^{-1})'\II(\sqrt{|\AA|}\CC^{-1})'
		\end{aligned}
		\end{equation}
		即$\AA\backsimeq\II$
	\end{proof}
	\begin{exercise}
		设$n$阶方阵$\AA$满足条件$\AA'\AA=\II$,其中$\AA'$是$\AA$的转置矩阵,$\II$为单位矩阵,证明$\AA$的实特征向量所对应的特征值的绝对值等于1.
	\end{exercise}
	\begin{proof}
		已知$\AA'=\AA^{-1}$\\
		由于$|\lambda\II|=|(\lambda\II-\AA)'|=|\lambda\II-\AA'|=|\lambda\II-\AA^{-1}|=|\AA^{-1}||\lambda\II-\AA|=$
		\begin{equation}
		\frac{1}{|\AA|}\frac{1}{\lambda}(-1)^n|\lambda\II-\AA|
		\end{equation}
		所以$\frac{1}{|\AA|}\frac{1}{\lambda}(-1)^n=1$,而$|\AA\AA'|=|\AA|^2=1,$所以$|\lambda|=1$
	\end{proof}
	\begin{exercise}
		设$f(x_1,\cdots,x_n)=\XX^T\AA\XX$是一实二次型,若存在n维实向量
		\begin{equation}
		\XX_1,\XX_2,s.t.,\XX_1^T\AA\XX_1>0,\XX_2^T\AA\XX_2<0,
		\end{equation}
		证明:存在$n$维实向量$\AA,s.t.,\AA'\AA\AA=0.$
	\end{exercise}
	\begin{proof}
		设$\AA$的秩为$r$,作实非退化线性替换$\XX=\CC\YY$,将$f$化为规范形
		\begin{equation}
		f(x_1,\cdots,x_n)=\XX'\AA\XX=y_1^2+\cdots+y_p^2-y^2_{p+1}-\cdots-y^2_{p+q}(r=p+q)
		\end{equation}
		由于存在两个向量$\XX_1,\XX_2$,使$\XX'_1\AA\XX_1>0,\XX'_2\AA\XX_2<0$,,从而可得$p>0,q>0$.\\
		令
		\begin{equation}
		y_1=y_{p+1}=1,y_2=\cdots=y_p=y_{p+2}=\cdots=y_{p+q}=0,
		\end{equation}
		取$n$维列向量
		\begin{equation}
		\XX_0=\CC
		\begin{bmatrix}
		1\\0\\\vdots\\0\\1\\0\\\vdots\\0
		\end{bmatrix}
		\end{equation}
		中间的1位于第$p+1$行.\\
		由于$\XX=\CC\YY$非退化,故$\XX_0\ne\boldsymbol{0}$,且有
		\begin{equation}
		\XX'_0\AA\XX_0=0
		\end{equation}
	\end{proof}
	\begin{exercise}
		设$\AA$是$n$阶实对称矩阵,证明:
		\begin{equation}
		\exists c>0,s.t., \forall \XX\in \boldsymbol{R}^n,|\XX^T\AA\XX|\leq c\XX^T\XX.
		\end{equation}
	\end{exercise}
	\begin{proof}
		设$\AA$的全部特征值是$\lambda_1,\cdots,\lambda_n$.则$t\II+\AA$的全部特征值为$t+\lambda_1,\cdots,t+\lambda_n$,当$t>S_r(\AA)$时.
		\begin{equation}
		t+\lambda_i\geq t-|\lambda_i|\geq t-S_r(\AA)>0,\;i=1,\cdots,n.
		\end{equation}
		即$t\II+\AA$为正定矩阵.同理$t\II-\AA$也为正交矩阵.\\
		任意一个实n阶向量$\XX$,都有.
		\begin{equation}
		\XX'(c\II+\AA)\XX\geqslant 0,\;\XX'(c\II-\AA)\geqslant 0
		\end{equation}
		\begin{equation}
		-c\XX'\XX\leqslant \XX'\AA\XX\leqslant c\XX'\XX
		\end{equation}
		即
		\begin{equation}
		|\XX'\AA\XX|\leqslant c\XX'\XX
		\end{equation}
	\end{proof}
	\begin{exercise}
		(1)证明:如果${\displaystyle \sum^n_{i=1}\sum^n_{j=1}}a_{ij}x_ix_j(a_{ij}=a_{ji})$是正定二次型.那么
		\begin{equation}
		f(y_1,\dots,y_n)=
		\begin{bmatrix}
		a_{11}&a_{12}&\cdots&a_{1n}&y_1\\
		a_{21}&a_{22}&\cdots&a_{2n}&y_2\\
		\vdots&\vdots&\ddots&\vdots&\vdots\\
		a_{n1}&a_{n2}&\cdots&a_{nn}&y_n\\
		y_1&y_2&\cdots&y_n&0
		\end{bmatrix}
		\end{equation}
		是负定二次型;\\
		(2)如果$\AA=
		\begin{bmatrix}
		a_{11}&a_{12}&\cdots&a_{1n}\\
		a_{21}&a_{22}&\cdots&a_{2n}\\
		\vdots&\vdots&\ddots&\vdots\\
		a_{n1}&a_{n2}&\cdots&a_{nn}
		\end{bmatrix}
		$是正定矩阵,那么
		\begin{equation}
		|\AA|\leq a_{nn}\PP_{n-1},
		\end{equation}
		其中$\PP_{n-1}$是$\AA$的n-1阶顺序主子式;\\
		(3)$|\AA|\leq a_{11}a_{22}\cdots a_{nn}.$
	\end{exercise}
		\begin{proof}
		(1)作变换$\YY=\AA\ZZ$即
		\begin{equation}
		\begin{bmatrix}
		y_1\\\vdots\\y_n
		\end{bmatrix}
		=
		\begin{bmatrix}
		a_{11}&\cdots&a_{1n}\\
		\vdots&      &\vdots\\
		a_{n1}&\cdots&a_{nn}
		\end{bmatrix}
		\begin{bmatrix}
		z_1\\\vdots\\z_n
		\end{bmatrix}
		\end{equation}
		则
		\begin{equation}
		\begin{aligned}
		f(y_1,\cdots,y_n)&=
		\begin{vmatrix}
		a_{11}&\cdots&a_{1n}&a_{11}z_1+\cdots+a_{1n}z_n\\
		\vdots&&\vdots&\vdots\\
		a_{n1}&\cdots&a_{nn}&a_{n1}z_1+\cdots+a_{nn}z_n\\
		y_1&\cdots&y_n&0
		\end{vmatrix}\\
		&\xRightarrow[(i=1,\cdots,n-1)]{c_n-z_nc_i}
		\begin{vmatrix}
		a_{11}&\cdots&a_{1n}&0\\
		\vdots&&\vdots&\vdots\\
		a_{n1}&\cdots&a_{nn}&0\\
		y_1&\cdots&y_n&-(y_1z_1+\cdots y_nz_n)
		\end{vmatrix}\\
		&=-|\AA|(y_1z_1+\cdots y_nz_n)\\
		&=-|\AA|\YY'\ZZ=-|\AA|\ZZ'\AA\ZZ
		\end{aligned}
		\end{equation}
		由$\AA$为正定矩阵知,$f(y_1,y_2,\cdots,y_{n-1})$为负定二次型.\\
	\end{proof}
	\begin{proof}
		(2)记
		\begin{equation}
		g'(y_1,y_2,\cdots,y_n)=
		\begin{vmatrix}
		a_{11}&a_{12}&\cdots&a_{1,n-1}&y_1\\
		a_{21}&a_{22}&\cdots&a_{2,n-1}&y_2\\
		\vdots&\vdots&&\vdots&\vdots\\
		a_{n1}&a_{n2}&\cdots&a_{n,n-1}&y_n\\
		y_1&y_2&\cdots&y_{n-1}&0
		\end{vmatrix}
		\end{equation}
		由(1)可知,$g'(y_1,y_2,\cdots,y_{n-1})$负定
		\begin{equation}
		|\AA|=g'(a_{n1},\cdots,a_{n,n-1})+a_{nn}\PP_{n-1}
		\end{equation}
		由于$g'(a_{n1},\cdots,a_{n,n-1})<0$,所以$|\AA|\leq a_{nn}\boldsymbol{\cdot}\PP_{n-1}$.
	\end{proof}
	\begin{proof}
		(3)由(2),得
		\begin{equation}
		|\AA|\leq a_{nn},|\PP_{n-1}|\leq a_{nn}a_{n-1,n-1},|\PP_{n-2}|\leq \cdots\leq a_{nn}a_{n-1,n-1}\cdots a_{11}
		\end{equation}
	\end{proof}
	\begin{exercise}
		设$\AA,\BB$都是$n\times n$实对称矩阵,且$\AA$正定,\\
		(1)证明:存在实可逆矩阵$\TT$,使得$\TT'(\AA+\BB)\TT$为对角矩阵;\\
		(2)假设$\BB$也正定.证明:$|\AA+\BB|\geq|\AA|+|\BB|.$\\
		(3)假设$\BB$也正定,且$\AA\BB=\BB\AA$,证明$\AA\BB$也是正定矩阵.
	\end{exercise}
	\begin{proof}
		(1)\\
		$\AA$正定,所以存在正交矩阵$\PP^{-1}=\PP'$使得,
		\begin{equation}
		\PP'\AA\PP=diag\left(a_1,\cdots,a_n\right),
		\end{equation}
		$a_1,\cdots,a_n>0$为$\AA$的特征值.\\
		$\PP'\BB\PP=\PP^{-1}\BB\PP$也是对称矩阵,且与$\BB$有相同的特征值.所以存在正交矩阵$\QQ^{-1}=\QQ'$使得,
		\begin{equation}
		\QQ'(\PP'\BB\PP)\QQ=\QQ^{-1}(\PP^{-1}\BB\PP)\QQ=diag\left(b_1,\cdots,b_n\right),
		\end{equation}
		其中$b_1,\cdots,b_n$为$\BB$的特征值.所以:
		\begin{equation}
		\QQ'(\PP'(\AA+\BB)\PP)\QQ=\QQ^{-1}(\PP^{-1}(\AA+\BB)\PP)\QQ=diag\left(a_1+b_1,\cdots,a_n+b_n\right).
		\end{equation}
		所以存在实可逆矩阵$\TT=\PP\QQ$,使得$\TT'(\AA+\BB)\TT$为对角矩阵.
	\end{proof}
	\begin{proof}
		(2)\\
		$\BB$也正定,因此$b_1,\cdots,b_n>0$.
		\begin{align*}
			|\QQ'(\PP'(\AA+\BB)\PP)\QQ|
			&=|diag(a_1+b_1,\cdots,a_n+b_n)|\\
			&=\prod_{i=1}^{n}(a_i+b_i)\\
			&>\prod_{i=1}^{n}a_i+\prod_{i=1}^{n}b_i\\
			&=|\PP^{-1}\AA\PP|+|\PP^{-1}\BB\PP|=|\AA|+|\BB|.
			\end{align*}
			\begin{equation}
			\Rightarrow|\AA+\BB|\ge|\AA|+|\BB|.
			\end{equation}
	\end{proof}
	\begin{proof}
		(3)存在一个n级正交矩阵$\TT,s.t.$
		\begin{equation}
		\TT'\AA\TT=diag\{\lambda_1,\cdots,\lambda_n\}\quad\TT'\BB\TT=diag\{\mu_1,\cdots,\mu_n\}.
		\end{equation}
		其中$\lambda_1,\cdots,\lambda_n$是$\AA$的全部特征值,$\mu_1,\cdots,\mu_n$是$\BB$的全部特征值.$\AA,\BB$都是正定矩阵,由于:
		\begin{equation}
		\TT'(\AA\BB)\TT=(\TT'\AA\TT)(\TT'\BB\TT)=diag\{\lambda_1\mu_1,\cdots,\lambda_n\mu_n \}
		\end{equation}
		即$\AA\BB$合同与$diag\{\lambda_1\mu_1,\cdots,\lambda_n\mu_n \}$,所以$\AA\BB$是正定矩阵.
	\end{proof}
	\begin{exercise}
		化二次型
		\begin{equation}
		f(x_1,x_2,x_3)=2x_1^2+4x_1x_2-4x_1x_3+5x_2^2-8x_2x_3+5x_3^2
		\end{equation}
		为标准型并写出线性变换.
	\end{exercise}
	\begin{proof}
		二次型f的矩阵$\AA$,以及$\AA$的全部特征值以及特征向量为:
		\begin{equation}
		\AA=
		\begin{bmatrix}
		2&2&-2\\
		2&5&-4\\
		-2&-4&5
		\end{bmatrix};
		\end{equation}
		$\lambda_1,\lambda_2,\lambda_3$的特征值分别为10,1,1特征向量分别为
		\begin{equation}
		\boldsymbol{v}_1=\begin{bmatrix}-1\\-2\\2\end{bmatrix}\;\boldsymbol{v}_2=\begin{bmatrix}2\\0\\1\end{bmatrix}\;\boldsymbol{v}_3=\begin{bmatrix}-2\\1\\0\end{bmatrix}
		\end{equation}
		将$\boldsymbol{v}_1,\boldsymbol{v}_2,\boldsymbol{v}_3$正交化,单位化得$\bfbt_1,\bfbt_2,\bfbt_3$:
		\begin{equation}
		\bfbt_1=
		\begin{bmatrix}
		-\frac{1}{3}\\-\frac{2}{3}\\\frac{2}{3}
		\end{bmatrix}
		\bfbt_2=
		\begin{bmatrix}
		\frac{2}{\sqrt{5}}\\0\\\frac{1}{\sqrt{5}}
		\end{bmatrix}
		\bfbt_3=
		\begin{bmatrix}
		-\frac{2}{3\sqrt{5}}\\\frac{5}{3\sqrt{5}}\\\frac{4}{3\sqrt{5}}
		\end{bmatrix}
		\end{equation}
		即
		\begin{equation}
		\TT=
		\begin{bmatrix}
		-\frac{1}{3}&\frac{2}{\sqrt{5}}&-\frac{2}{3\sqrt{5}}\\
		-\frac{2}{3}&0&\frac{5}{3\sqrt{5}}\\
		\frac{2}{3}&\frac{1}{\sqrt{5}}&\frac{4}{3\sqrt{5}}
		\end{bmatrix}
		\qquad\TT'\AA\TT=\DD=diag\left(10,1,1\right);
		\end{equation}
		令
		\begin{equation}
		\begin{bmatrix}
		x_1\\x_2\\x_3
		\end{bmatrix}
		=
		\TT
		\begin{bmatrix}
		y_1\\y_2\\y_3\
		\end{bmatrix}.
		\end{equation}
		则
		\begin{equation}
		f(y_1,y_2,y_3)=10y_1^2+y_2^2+y_3^2.
		\end{equation}
	\end{proof}
	\begin{exercise}
		设二次型
		\begin{equation}
		f(x_1,x_2,x_3)=ax_1^2+ax_2^2+(a-1)x_3^2+2x_1x_3-2x_2x_3.
		\end{equation}
		(1)求二次型$f$的矩阵的所有特征值;\\
		(2)若二次型的规范性为$y_1^2+y_2^2$,求a的值和各特征值的一个特征向量.
	\end{exercise}
	\begin{proof}
		(1)二次型的矩阵$\AA$为:
		\begin{equation}
		\AA=
		\begin{bmatrix}
		a&0&1\\
		0&a&-1\\
		1&-1&a-1
		\end{bmatrix}\;
		|\lambda\II-\AA|=
		\begin{vmatrix}
		\lambda-a&0&-1\\
		0&\lambda-a&1\\
		-1&1&\lambda-(a-1)
		\end{vmatrix}
		\end{equation}
		解得$\lambda_1,\lambda_2,\lambda_3$的特征值分别为$a-2,a,a+1$,特征向量分别为
		\begin{equation}
		\boldsymbol{v}_1=\begin{bmatrix}-1\\1\\2\end{bmatrix},\;\boldsymbol{v}_2=\begin{bmatrix}1\\1\\0\end{bmatrix},\;\boldsymbol{v}_3=\begin{bmatrix}1\\-1\\1\end{bmatrix}.
		\end{equation}
	\end{proof}
	\begin{proof}
		(2)若规范性为$y_1^2+y_2^2,a$只能为$2$,即:\\
		$\lambda_1,\lambda_2,\lambda_3$的特征值分别为$0,2,3$,特征向量分别为
		\begin{equation}
		\boldsymbol{v}_1=\begin{bmatrix}-1\\1\\2\end{bmatrix},\;\boldsymbol{v}_2=\begin{bmatrix}1\\1\\0\end{bmatrix},\;\boldsymbol{v}_3=\begin{bmatrix}1\\-1\\1\end{bmatrix}.
		\end{equation}
	\end{proof}
	\begin{exercise}
		若$n\times n$矩阵$\AA$可逆,\\
		(1)证明$\AA'\AA$正定.\\
		(2)证明:存在正交矩阵$\PP,\QQ$,使得
		\begin{equation}
		\PP'\AA\QQ=diag\left(a_1,\cdots,a_n\right),where \; a_i>0,i=1,\cdots,n.
		\end{equation}
	\end{exercise}
	\begin{proof}
		(1)
		\begin{equation}
		\AA'\AA=\AA'\II\AA
		\end{equation}
		因此$\AA'\AA$合同与单位矩阵,即$\AA'\AA$正定.
	\end{proof}
	\begin{proof}
		(2)
		$\AA'\AA$正定,因此存在正交矩阵$\PP$,使得
		\begin{equation}
		\PP'\AA'\AA\PP=diag(b_1,\cdots,b_n),where\;b_i>0,i=0,\cdots,n.
		\end{equation}
		令
		\begin{equation}
		\DD=diag\left(\sqrt{b_1},\cdots,\sqrt{b_n}\right),\;\QQ=\AA'\PP\DD,
		\end{equation}
		容易求得
		\begin{equation}
		\QQ'\QQ=(\AA'\PP\DD)'\AA'\PP\DD=\II_n,
		\end{equation}
		因此矩阵$\QQ$为正交矩阵,令$a_i=\sqrt{b_i},i=1\cdots,n.$则
		\begin{equation}
		\PP'\AA\QQ=\DD=diag\left(a_1,\cdots,a_n\right).
		\end{equation}
	\end{proof}
	\begin{exercise}
		求线性方程组
		$
		\begin{cases}
		2x_1+x_2+3_3x_3-x_4=0\\
		3x_1+2x_2+0x_3-2x_4=0\\
		3x_1+x_2+9x_3-x_4=0
		\end{cases}
		$的解空间$W$及其正交补空间$W^{\perp}$.
	\end{exercise}
	\begin{proof}
		系数矩阵作初等行变换得:
		\begin{equation}
		\AA=
		\begin{bmatrix}
		2&1&3&-1\\
		3&2&0&-2\\
		3&1&9&-1
		\end{bmatrix}
		\xRightarrow{}
		\begin{bmatrix}
		1&0&6&0\\
		0&1&-9&-1\\
		0&0&0&0
		\end{bmatrix}
		\end{equation}
		因此,方程组的两个线性无关的向量为:
		\begin{equation}
		\bfaf_1=
		\begin{bmatrix}
		0\\1\\0\\1
		\end{bmatrix},\;
		\bfaf_2=
		\begin{bmatrix}
		-6\\9\\1\\0
		\end{bmatrix}.
		\end{equation}
		$W=L(\bfaf_1,\bfaf_2)$.
		设正交补空间$W^\perp$的任一向量为$\bfbt=(y_1,y_2,y_3,y_4)'$.则$\bfaf_1'\bfbt=\bfaf_2'\bfbt=0$.即
		\begin{equation}
		\begin{cases}
		y_2+y_4=0\\
		-6y_1+9y_2+y_3=0
		\end{cases}
		\end{equation}
		因此,上面的方程组的两个线性无关的向量为:
		\begin{equation}
		\bfbt_1=
		\begin{bmatrix}
		1\\0\\6\\0
		\end{bmatrix},\;
		\bfbt_2=
		\begin{bmatrix}
		-3\\-2\\0\\2
		\end{bmatrix}.
		\end{equation}
		所以,正交补空间$W^\perp=L(\bfbt_1,\bfbt_2)$.
	\end{proof}
	\begin{exercise}
		设$\sigma$是有限维线性空间$V$的可逆线性变换.设$W$是$V$中$\sigma-$不变子空间,证明:$W$是$V$中$\sigma^{-1}-$不变子空间.
	\end{exercise}
	\begin{proof}
		已知$\bfaf\in W,\sigma(\bfaf)\in W$.\\
		设$\sigma(\bfaf)=\bfbt$,因$\sigma$可逆,即若:
		\begin{equation}
		\sigma^{-1}(\bfbt)=\bfaf\qquad\bfbt\;and\;\sigma^{-1}(\bfbt)\in W
		\end{equation}
		所以$W$是$V$中$\sigma^{-1}-$不变子空间
	\end{proof}
	下证$\sigma$为双射:
	\begin{proof}
		设$\sigma$为可逆变换,它的逆变换为$\sigma^{-1}$.\\
		任取$\bfxi,\bfeta\in W$,且$\bfxi\ne\bfeta$,则必有$\sigma(\bfxi)\ne\sigma(\bfeta)$.不然,设$\sigma(\bfxi)=\sigma(\bfeta)$,两边左乘$\sigma^{-1}$,有$\bfxi=\bfeta$,这与条件矛盾,所以$\sigma$是单射.\\
		其次,对任一向量$\bfxi\in W$,必有$\bfeta$使$\sigma(\bfxi)=\bfeta$,事实上,令$\sigma^{-1}(\bfeta)=\bfxi$即可,所以$\sigma$是满射,故$\sigma$是双射,即$\sigma^{-1}(\bfbt)=\bfaf$.
	\end{proof}

	\begin{exercise}
		设$\sigma$是正交变换,证明:$\sigma$的不变子空间的正交补也是$\sigma$的不变子空间.
	\end{exercise}
	\begin{proof}
		设$W$是$\sigma$的任意一个不变子空间,取$W$的一组标准正交基$\varep_1,\cdots,\varep_n$,把它扩成$V$的一组标准正交基$\varep_1,\cdots,\varep_m,\varep_{m+1},\cdots,\varep_n$,由
		\begin{equation}
		dim(W)+dim(W^\perp)=n
		\end{equation}
		\begin{equation}
		W=L(\varep_1,\cdots,\varep_m),\;W^\perp=L(\varep_{m+1},\cdots,\varep_n).
		\end{equation}
		又$\sigma$是正交变换,则$\sigma(\varep_1),\cdots,\sigma(\varep_n)$也是标准正交基,由于$W$是$\sigma$的不变子空间,所以$\sigma(\varep_1),\cdots,\sigma(\varep_m)\in W$.且为$W$的一组标准正交基,从而$\sigma(\varep_{m+1},\cdots,\sigma(\varep_n))\in W^\perp$\\
		任取$\alpha\in W^\perp$则
		\begin{equation}
		\alpha=k_{m+1}\varep_{m+1}+\cdots+k_n\varep_n\in W^\perp,
		\end{equation}
		所以
		\begin{equation}
		\sigma(\alpha)=k_{m+1}\varep_{m+1}+\cdots+k_n\sigma(\varep_n)\in W^\perp.
		\end{equation}
		$W^\perp$是$\sigma$的不变子空间.
	\end{proof}
	\begin{exercise}
		证明:反对称矩阵的特征值为0或者纯虚数。
	\end{exercise}
	\begin{proof}
		设$\AA$是反对称实矩阵,$\lambda$是$\AA$的一个特征值,$\bfxi$为相应的特征向量,即$\AA\bfxi=\lambda\bfxi$,则
		\begin{equation}
		\bar{\bfxi}'\AA\bfxi=\bar{\bfxi}'(-\AA')\bfxi=-\bar{\bfxi}'\AA'\bfxi=-(\AA\bar{\bfxi})'\bfxi=-(\overline{\AA\bfxi})'\bfxi
		\end{equation}
		即有$\lambda\bar{\bfxi}\bfxi=-\bar{\lambda}\bar{\bfxi}\bfxi$,从而$\lambda=-\bar{\lambda}$.\\
		令$\lambda=a+ib$,代入上式得$a+ib=-(a-ib)$,即有$a=0$.故$\lambda$是0或纯虚数.
	\end{proof}
	\begin{exercise}
		如果$\lambda$是正交矩阵$\AA$的特征值,则$\frac{1}{\lambda}$也是正交矩阵$\AA$的特征值.
	\end{exercise}
	\begin{proof}
		设$\lambda$是$\AA$的特征值,已知$\lambda^{-1}$是$\AA^{-1}$的特征值,由于$\AA$是正交矩阵,$\AA'=\AA^{-1}$,所以$\lambda^{-1}$是$\AA'$的特征值.因为$\AA$与$\AA'$有相同的特征值,所以$\lambda^{-1}$也是$\AA$的特征值.
	\end{proof}
	\begin{exercise}
		设$\AA,\BB$都是实对称矩阵,证明:存在正交矩阵$\QQ,s.t.,\QQ^{-1}\AA\QQ=\BB$的充要条件为$\AA,\BB$的特征多项式的根全部相同.
	\end{exercise}
	\begin{proof}
		必要性.若$\TT'\AA\TT=\BB$,即$\AA,\BB$相似,则$\AA,\BB$的特征多项式相同,所以它们的特征根相同.\\
		充分性.设$\lambda_1,\lambda_2,\cdots,\lambda_n$为$\AA$和$\BB$的特征值,由$\AA,\BB$为实对称矩阵,则存在正交矩阵$\XX$和$\YY$,使
		\begin{equation}
		\XX^{-1}\AA\XX=
		\begin{bmatrix}
		\lambda_1&&\\
		&\ddots&\\
		&&\lambda_n
		\end{bmatrix}
		=\YY^{-1}\BB\YY
		\end{equation}
		于是
		\begin{equation}
		\YY\XX^{-1}\AA\XX\YY^{-1}=\BB
		\end{equation}
		令$\TT=\XX\YY^{-1}$,由$\XX,\YY$是正交矩阵知,$\TT$也是正交矩阵,且有$\TT^{-1}\AA\TT=\BB$.
	\end{proof}
	\begin{exercise}
		设$\AA$是$n$阶实对称矩阵,且$\AA^2=\AA$.证明:存在正交矩阵
		\begin{equation}
		\QQ,s.t.,\QQ^{-1}\AA\QQ=diag\left(1,\cdots,1,0,\cdots,0\right).
		\end{equation}
	\end{exercise}
	\begin{proof}
		设$\lambda$是$\AA$的任意特征值,$\bfxi$是属于$\lambda$的特征向量,则$\AA\bfxi=\lambda\bfxi$,从而
		\begin{equation}
		\AA^2\bfxi=\AA(\AA\bfxi)=\AA(\lambda\bfxi)=\lambda\AA\bfxi=\lambda^2\bfxi.
		\end{equation}
		又由于$\AA^2=\AA$,所以有$\lambda^2\bfxi=\AA^2\bfxi=\AA\bfxi=\lambda\bfxi$,即$(\lambda^2-\lambda)\bfxi=0$,因为$\bfxi\ne\boldsymbol{0}$,故$\lambda^2-\lambda=0$,从而$\lambda=0$或1,再由定理7知,存在正交矩阵$\QQ$,使
		\begin{equation}
		\QQ^{-1}\AA\QQ=diag\left(1,\cdots,1,0,\cdots,0\right).
		\end{equation}
		在上式中的对角线元素中,1的个数为$\AA$的特征值1的个数,0的个数为$\AA$的特征值0的个数.
	\end{proof}
\end{document}
