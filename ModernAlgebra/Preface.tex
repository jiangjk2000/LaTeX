\chapter{修订版前言}\label{prf:1}
这本讲义《近世代数引论》在中国科学技术大学使用了十余年.学生的反映尚好,
主要问题是习题较难.所以我们在书末对于较难的习题增加了提示,同时也补充
了一些新的习题.在内容上去掉“幂零群与可解群”和“$n \geq 5$次一般方程的
根式不可解性”两节,把它们改成两个附录.如果学时不够,正文中某些部分也可
略去不讲或者只做简要介绍.例如伽罗瓦理论可以只介绍基本定理的内容和它的
应用,而不讲证明.对于希洛夫定理、有限生成阿贝耳群的结构、唯一因子分解
整环等内容也可以类似地处理.\par

\chapter{前言}\label{prf:2}
近世代数是讲述群、环、域(以及模)等代数对象基本性质的一门大学课程.它
是今后学习和研究代数学的基础,也是研究其他数学、物理学和计算机科学等不
可缺少的工具.\par
本书是我们于1982年在中国科技大学授课讲义基础上,经过五年教学实践改写而
成.原讲义共五章,为了在一学期(周四学时)内讲完,这次删去了模论和线性
代数两章.\par
近世代数从它产生的年代起就明显有别于古典代数学.它的主要研究对象不是代
数结构中的元素特性,而是各种代数结构本身和不同代数结构之间的相互联系
(同态).掌握近世代数中所体现的丰富的数学思想和方法,比背诵一些代数学
定义和名词字典要重要得多.我们在教学中几乎用半个学期讲述第一章群论,这
是因为在群论中体现了近世代数的基本研究思想和方法,而这些思想和方法在学
生过去学习中是不熟悉的.群论中的定理基本上可分为定量和定性两类:前者的
典型例子是拉格朗日定理,后者的典型例子是同态基本定理.我们着重讲授定性
内容,特别是同态基本定理和群在集合上的作用,这是群论的关键所在.\par
第二章讲述环论和域论初步,正文中的内容是标准的.但是在几个附录中,我们
介绍了在数学发展中有历史意义的几个课题(高斯二平方和问题,代数基本定理
,尺规作图,三等分角等),最后一章向学生展示关于域的有限伽罗瓦扩张的优
美理论.\par
最后,我们向过去几年里对此书的前身提供意见的许多学者、教师和学生表示深
深的谢意,我们也欢迎大家今后对此书给予更多的批评和指正.\par